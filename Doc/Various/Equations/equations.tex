\documentclass{report} 

%-------------------------
% begin and end equations
%-------------------------
\newcommand{\be} {\begin{equation}}
\newcommand{\ee} {\end{equation}}
\newcommand{\bea}{\begin{eqnarray}}
\newcommand{\eea}{\end{eqnarray}}
\newcommand{\ba} {\begin{array}}
\newcommand{\ea} {\end{array}}

%-----------------
% text formatting 
%-----------------
\newcommand{\noi}{\noindent}

%---------------
% special signs
%---------------
\newcommand{\p}  {\partial}
\newcommand{\s}  {\star}
\newcommand{\Dx} {\Delta}

%-----------------
% velocity vector
%-----------------
\newcommand{\uvw}{{\bf u}}          

%---------------------------
% putting things up or down
%---------------------------
\newcommand{\ol} {\overline}
\newcommand{\ul} {\underline}

%---------------------
% logical coordinates
%---------------------
\newcommand{\ijk}     { {i,j,k} }      % central

\newcommand{\ipjk}    { {i+1,j,k} }    % i+1
\newcommand{\imjk}    { {i-1,j,k} }    % i-1
\newcommand{\ijpk}    { {i,j+1,k} }    % j+1
\newcommand{\ijmk}    { {i,j-1,k} }    % j-1
\newcommand{\ijkp}    { {i,j,k+1} }    % k+1
\newcommand{\ijkm}    { {i,j,k-1} }    % k-1

\newcommand{\ijpkp}   { {i,j+1,k+1} }  % i
\newcommand{\ijpkm}   { {i,j+1,k-1} }
\newcommand{\ijmkp}   { {i,j-1,k+1} }
\newcommand{\ijmkm}   { {i,j-1,k-1} }

\newcommand{\ipjkp}   { {i+1,j,k+1} }  % j
\newcommand{\ipjkm}   { {i+1,j,k-1} }
\newcommand{\imjkp}   { {i-1,j,k+1} }
\newcommand{\imjkm}   { {i-1,j,k-1} }

\newcommand{\ipjpk}   { {i+1,j+1,k} }  % k
\newcommand{\ipjmk}   { {i+1,j-1,k} }
\newcommand{\imjpk}   { {i-1,j+1,k} }
\newcommand{\imjmk}   { {i-1,j-1,k} }

%--------------------------------------
% compute length for framing equations
%--------------------------------------
\newlength{\ii}
\setlength{\ii}{\textwidth}
\addtolength{\ii}{-2\fboxsep}
\addtolength{\ii}{-2\fboxrule}

% To frame an equation, use:
% \begin{center} \fbox{ \begin{minipage}{\ii} \be
% ... equation ... 
% \ee \end{minipage} } \end{center}

%-------------------
% to include figures
%-------------------
\usepackage{graphicx}
\usepackage{psfrag}
% \newcommand{\thickbox}[2]{\thicklines\put(0,0){\framebox(#1,#2){}}}
\newcommand{\thickbox}[2]{}

%----------------------------
% fancy mathematical symbols
%----------------------------
\usepackage{amssymb}

%---------------
% PSI-Boil sign
%---------------
\newcommand{\psiboil}{\sffamily PSI-Boil}


\begin{document}  

%-------------------%
% Mass conservation %
%-------------------%
\section{Mass conservation}

In differential form:

\be
  \frac{\p \rho}{\p t} = - \nabla ( \rho \uvw )
\ee

In integral form:

\be
  \int_V \frac{\p \rho}{\p t} dV = - \int_S \rho \uvw dS
\ee

In the discrete form, they read:

\be
   \frac{\Delta V}{\Delta t} (\rho^N - \rho^{N-1}) 
   = 
   \Delta S (-\rho_w \uvw_w + \rho_e \uvw_e )
\ee

%-------------------------------%
% General conservation equation %
%-------------------------------%
\section{General conservation equation}

\be
  \frac{\p \rho \phi}{\p t} 
  = 
  - \nabla ( \rho \uvw \phi) 
  + \nabla (\Gamma \nabla \phi)
\ee

\be
  \int_V \frac{\p \rho \phi}{\p t} dV 
  = 
  - \int_S (\rho \uvw \phi) dS 
  + \int_S (\Gamma \nabla \phi)
\ee

\be
  \frac{\Delta V}{\Delta t} (\rho^N - \rho^{N-1}) 
  = 
    \Delta S [-(\rho \uvw\phi)_w + (\rho \uvw \phi)_e ] 
  + \Delta S [( \Gamma \nabla \phi)_w - (\Gamma \nabla \phi)_e]
\ee

\be
  \frac{\Delta V}{\Delta t} (\rho^N - \rho^{N-1}) 
  = 
    \Delta S [-(\rho \uvw\phi)_w + (\rho \uvw \phi)_e ] 
  + \Delta S [\Gamma_w \frac{\phi_c - \phi_w}{\Delta_w} - 
              \Gamma_e \frac{\phi_e - \phi_c}{\Delta_e} ]
\ee

%------------------------%
% Navier-Stokes equation %
%------------------------%
\section{Navier-Stokes equation}

\be
  \frac{\p \rho \uvw}{\p t} 
  + \nabla (\rho \uvw \uvw)
  =
  - \nabla p
  + \nabla \mu (\nabla \uvw + \nabla \uvw^T)
  + \sigma \kappa {\bf n}
  \; \; \; \; [ \frac{kg}{m^2 s^2} ]
\ee

where $\sigma$ is surface tension in $[N/m]$ or $[kg/s^2]$.
Divided by $\rho$, they read:

\be
  \frac{\p \uvw}{\p t} 
  + \nabla (\uvw \uvw)
  =
  - \frac{\nabla p}{\rho}
  + \nabla \nu (\nabla \uvw + \nabla \uvw^T)
  + \frac{\sigma}{\rho} \kappa {\bf n}
  \; \; \; \; [ \frac{m}{s^2} ]
\ee

The term $\kappa {\bf n}$ is computed as:

\be
  \kappa {\bf n} 
  = 
  - \nabla \cdot \frac{\nabla \alpha}{|\nabla \alpha|} \nabla \alpha
 \; \; \; \; [ \frac{1}{m^2} ]
\ee

Navier-Stokes equations in their integral form (one used by PSI-Boil):

\bea
  \int_V \frac{\p \rho \uvw}{\p t} dV + \int_S (\rho \uvw \uvw) dS 
  =
  & - & \int_V \nabla p \, dV                       \\ \nonumber
  & + & \int_S \mu (\nabla \uvw + \nabla^T \uvw) dS \\
  & + & \int_V \sigma \kappa {\bf n} \, dV
  \; \; \; \; [ \frac{kg \, m}{s^2} = N ]
\eea

%-----------------------------------------------------------%
% Correction of conservation equations, as derived by Yohei %
%-----------------------------------------------------------%
\section{Correction of conservation equations, \\ 
         as derived by Yohei}

General conservation equations in their integral form:

\be
    \underbrace{\frac{\p (\rho \phi)}{\p t}}_{\mathit{(U)nsteady}}
  + \underbrace{ \nabla (\rho \phi \uvw) }_{\mathit{(A)dvection}}
  =
  r.h.s.
\ee

If derivatives of this form are taken:

\be
    \underbrace{\rho \frac{\p \phi}{\p t}}_{U1} +
    \underbrace{\phi \frac{\p \rho}{\p t}}_{U2} 
  + \underbrace{\phi \uvw \nabla \rho}_{A1}
  + \underbrace{\rho \uvw \nabla \phi}_{A2}
  + \underbrace{\rho \phi \nabla \uvw}_{A3}
  =
  r.h.s.
\ee

If re-arranged:

\be
    \underbrace{\rho \left[ \frac{\p \phi}{\p t} + \uvw \nabla \phi\right]}_{U1+A2}
  + \underbrace{\phi \left[\frac{\p \rho}{\p t}          
  +               \uvw \nabla \rho           
  +             \rho \nabla \uvw \right]}_{U2+A1+A3}
  =
  r.h.s.
\ee

or:

\be
    \underbrace{\rho \left[ \frac{\p \phi}{\p t} + \uvw \nabla \phi\right]}_{U1+A2}
  + \underbrace{\phi \left[\frac{\p \rho}{\p t} + \nabla(\rho \uvw)\right]}_{U2+A1+A3=0}
  =
  r.h.s.
\ee

The second term ($U2+A1+A3$) dissapears beacause of the mass conservation. What 
remains to be solved, if we want to stay with conservative form of momentum 
conservation equations, is:

\be
    \underbrace{\rho \left[ \frac{\p \phi}{\p t} + \uvw \nabla \phi\right]}_{\mathit{To \, be \, solved}}
  =
  r.h.s.
\ee

What is actually solved in PSI-Boil is:

\be
    \underbrace{\rho \left[ \frac{\p \phi}{\p t} + \nabla (\uvw \phi)\right]}_{\mathit{Solved}}
  =
  r.h.s.
\ee

If the solved version is developed:

\be
    \rho \left( \frac{\p \phi}{\p t} 
  + \uvw \nabla \phi
  + \underbrace{\phi \nabla \uvw}_{\mathit{Additional}}\right)
  =
  r.h.s.
\ee

we see that an additional term in the equations appears. Hence, in order to be 
consistent with the conservative form of the momentum conservation equations,
we should discretize momentum equations as follows:

\be
    \rho \left[ \frac{\p \phi}{\p t} + \nabla (\uvw \phi) - \underbrace{\phi \nabla{\uvw}}_{\mathit{Correction}} \right]
  =
  r.h.s.
\ee

%---------------------------%
% Pressire-Poisson equation %
%---------------------------%
\section{Pressure-Poisson equation}

\be
  \frac{\nabla^2 p'}{\rho} = \frac{\nabla \uvw}{\Delta t}
  \; \; \; \; [ \frac{1}{s^2} ]
\ee

In integral form:

\be
  \int_S \frac{\nabla p'}{\rho} \, dS = \frac{1}{\Delta t} \int_S \uvw \, dS
  \; \; \; \; [ \frac{m^3}{s^2} ],    
\ee

or should it be:

\be
  \frac{1}{\rho} \int_S \nabla p' \, dS = \frac{1}{\Delta t} \int_S \uvw \, dS
  \; \; \; \; [ \frac{m^3}{s^2} ] \, ?
\ee

%---------------------%
% Velocity correction %
%---------------------%
\section{Velocity correction}

\be
  \uvw = \uvw + \frac{1}{\rho} \nabla p' \, \Delta t
\ee

or:

\be
  \uvw = \uvw + \nabla (\frac{p'}{\rho}) \, \Delta t
\ee

%------------------------%
% Fractional step method %
%------------------------%
\section{Fractional step method}

\subsection{In non-dimensional, tensorial form}

It is quite clearly explained in Despotis \& Tsangaris. Governing equations for 
viscous incompressible flow, written in non-dimensional form are re-written as:
%
\be
  \frac{u_i^\s-u_i^n}{\Delta t} 
  + 
  \frac{\p}{\p x_j} \left( u_j^n u_i^\s \right)
  =
  \frac{1}{Re} \frac{\p^2 u_i^\s}{\p x_j \p x_j}
\ee

\be
  \frac{u_i^{n+1} - u_i^\s}{\Delta t} 
  + 
  \frac{\p p^{n+1}}{\p x_i}
  =
  0
  \label{eq:step_2}
\ee
%
Note that these equations are written out for the fully implicit 
time-discretization. Pressure in Eq.~\ref{eq:step_2} is the full pressure,
not the pressure correction. 
If we take the divergence of equation~\ref{eq:step_2}, we get:
%
\be
  \frac{1}{\Delta t} 
    \left( 
      \underbrace{   \frac{\p u_i^{n+1}}{\p x_i} }_{=0} 
                   - \frac{\p u_i^\s}   {\p x_i} 
    \right) 
  + 
  \frac{\p^2 p^{n+1}}{\p x_j \p x_j}
  =
  0
  \label{eq:step_3}
\ee
%
Meaning that pressure-Poisson equation for the pressure is:
%
\be
  \frac{\p^2 p^{n+1}}{\p x_j \p x_j}
  =
  \frac{1}{\Delta t}  \frac{\p u_i^\s}   {\p x_i} 
  \label{eq:step_4}
\ee
%
and new velocity field is (from~\ref{eq:step_2}):
%
\be
  u_i^{n+1} 
  =
  u_i^\s 
  - 
  \Delta t \frac{\p p^{n+1}}{\p x_i}
  \label{eq:step_5}
\ee
%

\subsection{In dimensional, vector, non-conservative form}

This section gives the same derivation, but for the vector form I used before.
Surface tension terms are ommitted for simplicity. Advection terms are 
treated with fully explicit and diffusion terms by fully implicit scheme.

\be
  \frac{\uvw^\s - \uvw^n}{\Delta t} 
  + \nabla (\uvw^n \uvw^n)
  =
  \nu \nabla^2 \uvw^\s
  \; \; \; \; [ \frac{m}{s^2} ]
\ee

\be
  \frac{\uvw^{n+1} - \uvw^\s}{\Delta t} 
  +
  \frac{\nabla p}{\rho}
  = 
  0
  \; \; \; \; [ \frac{m}{s^2} ]
\ee

Taking the divergence of the last equations, leads to:

\be
  \frac{1}{\Delta t} 
  \left(   \underbrace{ \nabla \uvw^{n+1} }_{=0}
                      - \nabla \uvw^\s \right)
  +
  \frac{\nabla^2 p}{\rho}
  = 
  0
  \; \; \; \; [ \frac{1}{s^2} ]
\ee

or:

\be
  \frac{\nabla^2 p}{\rho} = \frac{\nabla \uvw^\s}{\Delta t}
  \; \; \; \; [ \frac{1}{s^2} ]
\ee

Velocity is corrected from:

\be
  \uvw^{n+1} 
  =
  \uvw^\s
  +
  {\Delta t} \frac{\nabla p}{\rho}
  \; \; \; \; [ \frac{m}{s} ]
\ee

\subsection{In dimensional, vector, conservative form}

Finally, I will try to re-write the equations in conservative form.
This is going to be tricky. Very tricky. Actually, we will encounter
several major problems. 
%
\be
  \frac{\rho^\s \uvw^\s - \rho^n \uvw^n}{\Delta t} 
  + \nabla (\rho^n \uvw^n \uvw^n)
  =
  \nabla \left( \mu \nabla \uvw^\s \right)
  \; \; \; \; [ \frac{kg}{m^2 s^2} ]
\ee
%
{\bf First problem:} how much is $\rho^\s$? It could be obtained from
forward Euler approximation of mass conservation equation: 
$\rho^\s = \rho^n + \Delta t \nabla (\rho^n \uvw^n)$, but will it be
accurate enough? Will it be stable enough?

\be
  \frac{\rho^{n+1} \uvw^{n+1} - \rho^\s \uvw^\s}{\Delta t} 
  +
  \nabla p
  = 
  0
  \; \; \; \; [ \frac{kg}{m^2 s^2} ]
\ee
%
{\bf Second problem:} how much is $\rho^{n+1}$? Can it be computed in the
same way as $\rho^\s$? Is it the same? Can it be assumed to be the same?

Taking the divergence of the last equations, leads to:

\be
  \frac{1}{\Delta t} 
  \left(   \underbrace{ \nabla ( \rho^{n+1} \uvw^{n+1} ) }_{\neq 0}
                      - \nabla ( \rho^\s    \uvw^\s    ) \right)
  +
  \nabla^2 p
  = 
  0
  \; \; \; \; [ \frac{kg}{m^3 s^2} ]
\ee
%
with some with more difficulties. The underbraced term is not equal
to zero anymore, meaning the whole concept should be changed. Actually,
the underbraced term is: $\nabla(\rho^{n+1} \uvw^{n+1}) = ( \frac{\p \rho}{\p t} ) ^{n+1}$.

So, the pressure-Poisson equation should look like
\be
  \Delta t \nabla^2 p 
  = 
  \nabla (\rho^\s \uvw^\s)
  -
  ( \frac{\p \rho}{\p t} ) ^{n+1}
  \; \; \; \; [ \frac{kg}{m^3 s} ]
\ee
%
So how a final correction looks like, it is hard to guess at this point. 
I should either find a good source of information on that topic, or
abandon conservative discretization scheme. Afterall, there must be 
a good reasong why so many papers are relying on non-conservative
discretizations.

\subsection{In dimensional, vector, conservative form, again}

I was forgetting one importatn aspect in previous section. I was using
only momentum conservation equation, but not the mass conservations.
So, the system should read:

\be
  \frac{\p \rho}{\p t} = - \nabla ( \rho \uvw )
  \; \; \; \; [ \frac{kg}{m^3 s} ]
\ee

\be
  \frac{\p \rho \uvw}{\p t} 
  + \nabla (\rho \uvw \uvw)
  =
  - \nabla p
  + \nabla \mu (\nabla \uvw)
  \; \; \; \; [ \frac{kg}{m^2 s^2} ]
\ee

If we turn this system into sem-discretized form (again using forward
Euler for advection and backward Euler for diffusion), we get:

\be
  \frac{\rho^{n+1} - \rho^n}{\Delta t} = - \nabla ( \rho^n \uvw^n )
  \; \; \; \; [ \frac{kg}{m^3 s} ]
\ee

\be
  \frac{\rho^\s \uvw^\s - \rho^n \uvw^n }{\Delta t} 
  + \nabla (\rho^n \uvw^n \uvw^n)
  =
  + \nabla \mu (\nabla \uvw^\s)
  \; \; \; \; [ \frac{kg}{m^2 s^2} ]
\ee
%
Here we still have a problem of defining the~$\rho^\s$. We may work
around this by solving for $\rho^\s \uvw^\s$ rather than for $\uvw^\s$.
%
That should be followed by:
%
\be
  \frac{\rho^{n+1} \uvw^{n+1} - \rho^\s \uvw^\s}{\Delta t} 
  +
  \nabla p
  = 
  0
  \; \; \; \; [ \frac{kg}{m^2 s^2} ]
\ee
%
% % However, if the assumption~$\rho^\s = \rho^{n+1}$ holds, this 
% % reduces to:
% % \be
% %   \frac{\uvw^{n+1} - \uvw^\s}{\Delta t} 
% %   +
% %   \frac{\nabla p}{\rho^{n+1}}
% %   = 
% %   0
% %   \; \; \; \; [ \frac{m}{s^2} ]
% % \ee
% %
% % I am not sure if this is good. I mean, if it makes sense to continue down this road.
%
where $\rho^{n+1}$ is known. If we take the divergence, we get:
%
\be
  \nabla( \rho^{n+1} \uvw^{n+1} ) - \nabla ( \rho^\s \uvw^\s )
  +
  \Delta t \nabla^2 p
  = 
  0
  \; \; \; \; [ \frac{kg}{m^3 s} ]
\ee
%
which is a pressure-Poisson equation, but the first term on the 
left-hand side did not vanish. I mean, the unknown velocity $\uvw^{n+1}$
did not vanish. 

\subsection{Fractional step method by Zaleski}

These considerations are not new. In the paper from Lafaurie, Nardone,
Scardovelli and Zaleski, JCP~1994, they do the following:

\begin{enumerate}
  \item Enforce $\nabla \uvw^{n+1} = 0$ (their equation 18), 
        presumably to get rif of the unknown density at new time step.
  \item Set $\rho^\s = \rho^n$ during the whole projection step (their
        equations 28 - 31).
\end{enumerate}

In the paper Popinet and Zaleski, IJNMF 1999, they do the following:

\begin{enumerate}
  \item Enforce $\nabla \uvw^{n+1} = 0$ (their equation 22), 
        presumably to get rid of the density. 
  \item Solve equations for $(\rho \uvw)^\s$ (their equation 18), 
        presumably to get rid of the unknown density at the tentative
        time step. 
  \item Set $\rho^\s = \rho^{n+1}$ during the whole projection step (their
        equations 28 - 31), although I am not sure how this helps the
        matter.
\end{enumerate}

%------------%
% Dimensions % 
%------------%
\section{Dimensions}

\subsection{Dependent variables}

\begin{center}
  \begin{tabular*}{0.75\textwidth}{@{\extracolsep{\fill}}|ccc|} 
    \hline
         symbol   & description       & dimension                \\
    \hline
    \hline
         $\uvw$   & velocity          & $\frac{m}{s}$            \\
    \hline
         $p$      & pressure          & $Pa = \frac{kg}{ms^2} $  \\
    \hline
%        $q$      & heat flux         & $\frac{W}{m^2} $         \\
%   \hline
%        $T$      & temperature       & $K$                      \\
%   \hline
  \end{tabular*}
\end{center}

\subsection{Physical properties}

\begin{center}
  \begin{tabular*}{0.75\textwidth}{@{\extracolsep{\fill}}|cccc|} 
    \hline
         symbol    & description          & definition               & dimension         \\
    \hline
    \hline
%        c         & thermal capacity     &                          & $\frac{J}{kg K}$  \\
%   \hline
%        $\beta$   & volumetric expansion &                          & $\frac{1}{K}   $  \\
%   \hline
%        $\kappa$  & thermal diffusivity  & $\frac{\lambda}{\rho c}$ & $\frac{m^2}{s} $  \\
%   \hline
%        $\lambda$ & thermal conductivity &                          & $\frac{W}{mK}  $  \\
%   \hline
         $\mu$     & viscosity            &                          & $\frac{kg}{ms} $  \\
    \hline
         $\nu$     & kinematic viscosity  & $\frac{\mu}{\rho}$       & $\frac{m^2}{s} $  \\
    \hline
         $\rho$    & density              &                          & $\frac{kg}{m^3}$  \\
    \hline
         $\sigma$  & surface tension      &                          & $\frac{kg}{s^2}$  \\
    \hline
  \end{tabular*}
\end{center}

\end{document}
