\label{chap_single_phase}

This chapter deals with the solution of transport equation for incompressible
fluid flow. The equation which describes this phenomenon is momentum
conservation equation, given by~\ref{eq_momentum}, repeated here for
convenience:
%
\be
         \int_V \frac{\p \rho \uvw}{\p t} dV
       + \int_S \rho \uvw \uvw \, dS
       = \int_S \mu \nabla \uvw \, dS
       - \int_V \nabla p \, dV
       + {\bf F}
       \; \; \; \;
       [N]
  \label{eq_momentum_2}
\ee
%
This form, however,
poses a problem. We have one three equations (one for each velocity 
component), but for unknowns: velocity components and pressure. 

For incompressible flow problems, pressure has always been sought through
additional constraint, mass conservation equation:
%
\be
       \int_V \frac{\p \rho}{\p t} dV 
       = 
       - \int_S \rho \uvw d{\bf S}
       \; \; \; \;
       [\frac{kg}{s}]
  \label{eq_mass}
\ee
% 
Momentum and mass conservation have been linked in various ways, providing
different algorithms for computation of incompressible flows. The most
widely used in applied CFD are SIMPLE, SIMPLEC, SIMPLER and even PISO 
({\bf references}). For highly unsteady flow, at simulation of which
{\psiboil} aims, the most efficient is the {\em fractional step method},
sometimes referred to as the {\em projection} method. It combines momentum
conservation equation~(\ref{eq_momentum}) with mass conservation
equation~(\ref{eq_mass}), at intermediate time between two simulated
time steps, to give additional equation for pressure (already introduced
as~\ref{eq_pressure}):
%
\be
         \int_S \frac{\nabla p}{\rho} \, d{\bf S} 
       = \frac{1}{\Delta t} \int_S \uvw^\s \, d{\bf S}
       \; \; \; \; 
       [ \frac{m^3}{s^2} ],
  \label{eq_pressure_2}
\ee
%
often referred to as pressure-Poisson equation.

The fractional step method, which is used by {\psiboil}, is now outlined.
The momentum conservation equation~(\ref{eq_momentum_2}) must first be discretized
in time, part by part. The most usual approach is to use Crank-Nicolson
scheme for diffusion (to avoid very low time steps needed by the forward
Euler scheme) and to use Adams-Bashforth for convection, for the sake
of simplicity. Inertial terms are discretized with backward Euler scheme.
The time-discretized momentum equation then looks like:
%
\be
        {\bf I}^\s - {\bf I}^{n-1} 
       + \frac{3}{2} {\bf C}^{n-1} - \frac{1}{2} {\bf C}^{n-2}
       = \frac{1}{2} {\bf D}^\s + \frac{1}{2} {\bf D}^{n-1}
       + {\bf F}^{n-1}
  \label{eq_semi}
\ee
%
Here, $\s$ denotes the tentative (intermediate) time step, while 
${\bf I}^\s = \frac{1}{\Delta t} \int_V  \rho \uvw^\s dV$ 
and ${\bf D}^\s = \int_S \mu \nabla \uvw^\s \, d{\bf S}$ 
denote inertial and diffusive term in the tentative time step. 
Old (known) values of velocity (at time step $n-1$) define terms:
${\bf I}^{n-1} = \frac{1}{\Delta t} \int_V  \rho \uvw^{n-1} dV$,
and ${\bf D}^{n-1} = \int_S \mu \nabla \uvw^{n-1} \, d{\bf S}$, as well
as convective term at old time step: 
${\bf C}^{n-1} = \int_S \rho \uvw^{n-1}\uvw^{n-1} \, d{\bf S}$. Clearly,
${\bf C}^{n-2} = \int_S \rho \uvw^{n-2}\uvw^{n-2} \, d{\bf S}$.

Equation~\ref{eq_semi} is discretized in space giving a linear system
of equations for intermediate velocity~($\uvw^\s$). Once this
system is solved gives velocity field which does not, generally, 
satisfies mass conservation equation. It is used to compute pressure 
from~\ref{eq_pressure_2}. The computed pressure is used to {\em project}
velocity into a divergence free velocity field:
%
\be
  \uvw^{n} = \uvw^\s + \frac{1}{\rho} \nabla p \, \Delta t
  \label{eq_projection}
\ee
%
which represents velocity at new time step. 

% wrong: {\psiboil} uses only fractional step method to link velocity and pressure.
% wrong: This also means that {\psiboil} can be used to simulate only unsteady 
% wrong: phenomena. If steady solution is sought, the governing equations should
% wrong: be integrated long enough until steady state is reached.

