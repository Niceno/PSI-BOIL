\section{Obtaining the sources}
\label{sec_obtaining}

{\psiboil} is developed and supported for Linux operating systems.
The main reasons for that are free availability of {\tt C++} 
compilers\footnote{\tt www.thefreecountry.com/compilers/cpp.shtml},
{\tt MPI} libraries\footnote{\tt www-unix.mcs.anl.gov/mpi/mpich1, 
www.open-mpi.org}
{\tt cvs} program\footnote{\tt www.cvshome.org}, 
{\tt make} facility and {\tt autotools}.
% All these programs and libraries could be installed under Windows, but that
% requires significant effort which would eventually {\em go in vain}, since 
% {\psiboil} is intended to be run for large, three-dimensional simulations
% of unsteady flow phenomena which usually takes weeks or months of CPU time
% which is not conveniently performed on Windows-based PCs. From these reasons,
% there are no plans to port {\psiboil} to Windows. 
It can be compiled for Windows as well, but only under the Cygwin sub-system,
being a Linux-like environment for Windows. Cygwin comes with all tools and
libraries necessary to compile {\psiboil}, {\em except} {\tt MPI} libraries.

\subsection{Obtaining the {\psiboil} as tarball}

If you obtain {\psiboil} sources as a tarball, untar it in:

\begin{verbatim}
	> ~/Developemt/PSI-Boil
\end{verbatim}

Then you can skip the following parts in~\ref{sec_obtaining}, and jump to~\ref{sec_compiling}.

\subsection{Obtaining the {\psiboil} via CVS}

{\psiboil} is stored as a {\tt CVS} database on PSI's {\tt afs} system. Before you
try to obtain the sources, make sure that you have:

\begin{itemize}
  \item an account on PSI's {\tt afs} system,
  \item access to a Linux-based PC or workstation (computer,
        for short) {\em or} a Windows-based PC with Cygwin,
  \item {\tt CVS} installed on your system.
\end{itemize}

In most cases, you get an account on PSI's {\tt afs} system as soon as you
become PSI's employee, but in case you are not sure, or you are sure 
you do not have it, contact PSI's IT support person.

If you are sure you have an {\tt afs} account, you must also make sure you
have access to a Linux-based computer on which you want to compile 
and run {\psiboil}. That might be either your local PC, a remote cluster, 
or even a remote supercomputer (such as the ones at CSCS\footnote{\tt www.cscs.ch} 
in Manno). 
Remote computers can be accessed either from local PCs running Linux 
(easy and recommended) or Windows. In latter case, you may eiteher use
Cygwin sub-system, or a
program like {\tt putty\footnote{\tt www.putty.org}} and X-libraries for Windows 
(such as {\tt X-ming\footnote{\tt sourceforge.net/projects/xming}}) to
access the computer where you want to use {\psiboil}. Note that Cygwin can
be used as a replacement for {\tt putty + X-ming} to act as a terminal
to a Linux-based computer on which you compile and run, but it can also
be used as the environment for compiling and running under Windows itself. 

Once you access the Linux-based system on which you want to use {\psiboil},
it is very unlikely that it will not have a {\tt CVS} installation on it. 
To check it, try typing:
%
\begin{verbatim}
> cvs
\end{verbatim}
%
(followed by enter) in your shell and you should get something like:
%
{\small \begin{verbatim}
Usage: cvs [cvs-options] command [command-options-and-arguments]
  where cvs-options are -q, -n, etc.
    (specify --help-options for a list of options)
  where command is add, admin, etc.
    (specify --help-commands for a list of commands
     or --help-synonyms for a list of command synonyms)
  where command-options-and-arguments depend on the specific command
    (specify -H followed by a command name for command-specific help)
  Specify --help to receive this message

The Concurrent Versions System (CVS) is a tool for version control.
For CVS updates and additional information, see
    the CVS home page at http://www.cvshome.org/ or
    Pascal Molli's CVS site at http://www.loria.fr/~molli/cvs-index.html
\end{verbatim}}

\subsection{Adjusting the shell environment for {\tt CVS}}

If all of the above requirements are met, you are almost ready to get
the sources. However, the sources are retrieved by {\tt CVS} and it needs to
know where is the {\em repository}. {\tt CVS} is {\em instructed} by two 
environment variables: {\tt CVS\_RSH} and {\tt CVSROOT}. The formers instructs
{\tt CVS} which client program to use to access the repository and the letter
points to location of repository. These environment variables should be set 
in your shell initialization file. In case you use {\tt bash} shell, you should
edit the {\tt .bashrc} file and add the following two lines to it:

\begin{verbatim}
#-----#
# CVS #
#-----#
export CVS_RSH=ssh
export CVSROOT=:ext:your_afs_name@savannah.psi.ch:/afs/psi.ch/project/lth-sd/.CR
\end{verbatim}

where {\tt your\_afs\_name} has to be replaced with your {\tt afs} name, clearly. 
Once you add these lines into your {\tt .bashrc}, you should {\em source} it for 
the changes to take effect, using the following command:

\begin{verbatim}
> source ~/.bashrc
\end{verbatim}

Alternatively, you may log out and log in to your system again. 

\subsection{Retrieving the sources from {\tt CVS} database}

Once the environment variables for {\tt CVS} are set, you can go to directory
where you want {\psiboil} sources (for example: \verb"~"{\tt /Development} and get 
them with:
%
\begin{verbatim}
> cvs checkout PSI-Boil
\end{verbatim}
%
You will be asked for your password at {\tt savannah}:

\begin{verbatim}
your_afs_name@savanna.psi.ch's password:
\end{verbatim}

You should provide your {\tt afs} password at this point. When you do it, the
sources will start to appear in your directory. The output will look roughly 
like this:

{\small \begin{verbatim}
...
cvs checkout: Updating PSI-Boil/Src/Parallel
U PSI-Boil/Src/Parallel/Makefile.am
U PSI-Boil/Src/Parallel/communicator.cpp
U PSI-Boil/Src/Parallel/communicator.h
U PSI-Boil/Src/Parallel/mpi_macros.h
cvs checkout: Updating PSI-Boil/Src/Parallel/Out
U PSI-Boil/Src/Parallel/Out/Makefile.am
U PSI-Boil/Src/Parallel/Out/out.cpp
U PSI-Boil/Src/Parallel/Out/out.h
U PSI-Boil/Src/Parallel/Out/print.h
cvs checkout: Updating PSI-Boil/Src/Plot
U PSI-Boil/Src/Plot/Makefile.am
U PSI-Boil/Src/Plot/plot.cpp
U PSI-Boil/Src/Plot/plot.h
cvs checkout: Updating PSI-Boil/Src/Plot/GMV
U PSI-Boil/Src/Plot/GMV/Makefile.am
U PSI-Boil/Src/Plot/GMV/plot_gmv.cpp
U PSI-Boil/Src/Plot/GMV/plot_gmv.h
cvs checkout: Updating PSI-Boil/Src/Plot/TEC
U PSI-Boil/Src/Plot/TEC/Makefile.am
U PSI-Boil/Src/Plot/TEC/plot_tec.cpp
U PSI-Boil/Src/Plot/TEC/plot_tec.h
...
\end{verbatim}}

This source retrieval might take few minutes, depending on the speed of your
connection with {\tt savannah.psi.ch}. 


%---------------------------------------------------------------------nutshell-%
\vspace*{5mm} \fbox{ \begin{minipage}[c] {0.97\textwidth} %-----------nutshell-%
    {\sf Section \ref{sec_obtaining} in a nutshell} \\ %--------------nutshell-%

    - To obtain {\psiboil} sources via CVS, you need:
    %
    \begin{itemize}
      \item an account on PSI {\tt afs} system,
      \item access to a Linux-based computer or Windows-based PC with Cygwin,
            either with access to PSI's {\tt afs},
      \item {\tt CVS} software (installed on almost every Linux).
      \item environment variables {\tt CVS\_RSH} and {\tt CVSROOT} must be set
            properly.
    \end{itemize}

    -Provided all of the above is fulfilled, you get the sources with the
    command: 
    \begin{itemize}
      \item {\tt cvs checkout PSI-Boil}
    \end{itemize}
  \end{minipage} } %--------------------------------------------------nutshell-%
%---------------------------------------------------------------------nutshell-%



