\section{Obtaining the sources}
\label{sec_obtaining}

{\psiboil} is developed and supported for Linux operating systems.
The main reasons for that are free availability of {\tt C++} 
compilers\footnote{\tt www.thefreecountry.com/compilers/cpp.shtml},
{\tt MPI} libraries\footnote{\tt www-unix.mcs.anl.gov/mpi/mpich1, 
www.open-mpi.org}, 
{\tt make} facility and {\tt autotools}.
% All these programs and libraries could be installed under Windows, but that
% requires significant effort which would eventually {\em go in vain}, since 
% {\psiboil} is intended to be run for large, three-dimensional simulations
% of unsteady flow phenomena which usually takes weeks or months of CPU time
% which is not conveniently performed on Windows-based PCs. From these reasons,
% there are no plans to port {\psiboil} to Windows. 
It can be compiled for Windows as well, but only under the Cygwin sub-system,
being a Linux-like environment for Windows. Cygwin comes with all tools and
libraries necessary to compile {\psiboil}, {\em except} {\tt MPI} libraries.

\subsection{Prerequisites for obtaining the {\psiboil}}

{\psiboil} is stored in {\tt Github}  (https://github.com/Niceno/PSI-BOIL). 
Before you try to obtain the sources, make sure that you have:

\begin{itemize}
  \item an account on {\tt Github},
  \item access to a Linux-based PC or workstation (computer,
        for short) {\em or} a Windows-based PC with Cygwin,
  \item {\tt git} installed on your system.
\end{itemize}

To obtain the source code, type the next command:
\begin{verbatim}
	> git clone https://github.com/Niceno/PSI-BOIL
\end{verbatim}
%
(followed by enter) in your shell and you should get something like:
%
{\small \begin{verbatim}
		remote: Enumerating objects: 12801, done.
		remote: Counting objects: 100% (926/926), done.
		remote: Compressing objects: 100% (498/498), done.
		remote: Total 12801 (delta 447), reused 856 (delta 419), pack-reused 11875
		Receiving objects: 100% (12801/12801), 58.93 MiB | 21.76 MiB/s, done.
		Resolving deltas: 100% (8691/8691), done.
\end{verbatim}}

Then the folder {\tt PSI-BOIL} is created, which stores all the codes. 

\subsection{Usage of {\tt git}}

Please refer some books or ask {\tt ChatGPT} about the usage of {\tt Github}.  Here, several useful commands are listed. 

First, change directory to the folder: PSI-BOIL.
\begin{verbatim}
	> cd PSI-BOIL
\end{verbatim}

Then, to list all the branches of PSI-BOIL in {\tt Github},
\begin{verbatim}
	> git branch -a
\end{verbatim}

To show the current branch,
\begin{verbatim}
	> git branch --contains
\end{verbatim}

To switch to a branch, named e.g. branch\_to\_switch,

\begin{verbatim}
	> git checkout branch_to_switch
\end{verbatim}


To create a new branch (new\_branch) from a source branch (source\_branch),
\begin{verbatim}
	> git checkout -b new_branch old_branch
	 (if you have access right)
	> git push origin new_branch 
\end{verbatim}


%---------------------------------------------------------------------nutshell-%
\vspace*{5mm} \fbox{ \begin{minipage}[c] {0.97\textwidth} %-----------nutshell-%
    {\sf Section \ref{sec_obtaining} in a nutshell} \\ %--------------nutshell-%

    - To obtain {\psiboil} sources, you need:
    %
    \begin{itemize}
      \item an account on {\tt Github} system,
      \item access to a Linux-based computer or Windows-based PC with Cygwin,
            either with access to PSI's {\tt afs},
      \item {\tt git} software (installed on almost every Linux).
    \end{itemize}

    -Provided all of the above is fulfilled, you get the sources with the
    command: 
    \begin{itemize}
      \item {\tt git clone https://github.com/Niceno/PSI-BOIL}
    \end{itemize}
  \end{minipage} } %--------------------------------------------------nutshell-%
%---------------------------------------------------------------------nutshell-%



