\section{Compiling the sources}
\label{sec_compiling}

\subsection{Availability of required software}

In order to compile {\psiboil} you should obtain the sources first, which
is explained in~\ref{sec_obtaining}. If you don't have the sources yet, please
get them before proceeding with this section in the tutorial. Furthermore,
you need:
%
\begin{itemize}
   \item a {\tt C++} compiler and
   \item (hopefully) {\tt MPI} compiler and libraries.
\end{itemize}
%
{\tt C++} compiler is available on most Linux distributions, as well as 
{\tt MPI}. The most widely available on Linux is {\tt GNU C++} compiler. 
You can check if it is available on your system by running:
%
\begin{verbatim}
> g++
\end{verbatim}
%
{\psiboil} is mostly compiled with {\tt g++}, so it is a safe option 
to use it. 

To check if you have {\tt MPI} on your system, run the command:
%
\begin{verbatim}
> mpirun
\end{verbatim}
%
That should give you a hint whether MPI is installed, but also which {\em flavor}
it is. That is important to know. In case you had {\tt LAM/MPI} implementation,
the {\tt mpirun} command will result in output like:
%
{\small \begin{verbatim}
Synopsis:       mpirun [options] <app>
                mpirun [options] <where> <program> [<prog args>]

Description:    Start an MPI application in LAM/MPI.

Notes:
                [options]       Zero or more of the options listed below
...
\end{verbatim}}
%
Even if {\tt mpirun} command is available on your system, {\tt C++} compiler
for {\tt MPI} might not be. Try to run {\tt mpiCC} or {\tt mpicxx} in the shell.
%
If none of those is available, they might be available as modules. To check 
which modules are available on your system, us the command:
%
\begin{verbatim}
> module avail
\end{verbatim}
%
That should give an output like:
%
{\small \begin{verbatim}
----------------------------- /opt/Modules/modulefiles -----------------------------
acml/acml-3.1-pgi-64     intel-mkl/mkl-9.1_em64t  null
clhep/clhep-1.9.2.2      lam/lam-7.1.3            pgi/pgi-7.0_32
dot                      maple/10                 pgi/pgi-7.0_64
eclipse/europa           mathematica/5.2          pgi/pgi-7.1_32
fluent/Fluent-6.3.26     matlab/2007a_64          pgi/pgi-7.1_64
idl/idl-6.4              matlab/2008a             pgi/pgi-7.2_32
intel/intel-10.0_32      module-cvs               pgi/pgi-7.2_64
intel/intel-10.0_64      module-info              pgi/pgi_32-6.2
intel/intel-10.1_32      modules                  pgi/pgi_64-6.2
intel-mkl/mkl-10.0_32    mpi/mpich2-1.0.4-mpd     root/root-5.16.00
intel-mkl/mkl-10.0_64    mpi/mpich2-1.0.5         tecplot/tecplot-360
\end{verbatim}}
%
In the above case, two {\tt MPI} implementations are available on the system as modules,
namely the {\tt mpi/mpich2-1.0.4-mpd} and {\tt mpi/mpich2-1.0.5}. Load one of them,
for example:
%
\begin{verbatim}
> module load mpi/mpich2-1.0.4-mpd
\end{verbatim}
%
and try to find a compiler for {\tt MPI} again. If you can find {\tt mpi/openmpi}, then load it instead of {\tt mpi/mpich}.

If you do not have a {\tt C++} compiler, you can not continue. Ask your system 
administrator to install it for you. If you do not have a compiler for {\tt MPI} 
you might still compile a sequential version, although it is not recommended. 

\subsection{Creating {\tt Makefile}s}
\label{sub_sec_creating_makefiles}

Let's assume that you have {\tt C++} and {\tt MPI} and imagine that you ran 
{\tt cvs checkout PSI-Boil} command in directory {\tt ~/Development}.
In such a case, {\psiboil} will reside in:
%
\begin{verbatim}
~/Development/PSI-Boil
\end{verbatim}
%
and let's call this directory the {\em root} directory, for the sake of shortness.
(Just to make sure: inside the root directory, there are various 
sub-directories, such as: {\tt Benchmarks}, {\tt Src}, {\tt Doc}, etc.\ and 
files such as: {\tt AUTHORS}, {\tt NEWS}, {\tt README} \dots).

The creation of {\tt Makefile}'s for {\psiboil} is under control of 
{\tt autotools}. It is beyond the scope of this tutorial to describe {\tt autotools}, 
but there is a script residing in root directory, called {\tt first}, which
should be ran in order to create {\tt Makefile}s. It can be invoked with:
%
\begin{verbatim}
> ./first LAM
\end{verbatim}
%
to create {\tt Makefile}s for {\tt LAM/MPI} or:
%
\begin{verbatim}
> ./first MPICH
\end{verbatim}
%
if you use {\tt MPICH} libraries or:
%
\begin{verbatim}
	> ./first OPEN
\end{verbatim}
%
if you use {\tt OpenMPI} libraries. 

If you, unfortunately, do not a {\tt C++} compiler for {\tt MPI}, create {\tt Makefile}s with:
%
\begin{verbatim}
> ./first NOMPI
\end{verbatim}
%
(The difference between these three invocations
is merely a name of the {\tt MPI C++} compiler. In case of {\tt LAM/MPI} compiler
is called {\tt mpiCC}, whereas in the case of {\tt MPICH}, it is called {\tt mpicxx}.
If you have neither {\tt LAM/MPI} nor {\tt MPICH}, you can edit the scrip {\tt first}
to add more library, or platform options. At this moment, there is no appreciable 
difference between the two, you may chose whichever is more convenient for you.) 
In case if you choose the {\tt NOMPI} option, {\tt C++} compiler is set to {\tt GNU}'s
{\tt g++}. That should be installed on any Linux-based computer. 

No matter which of the above options you choose, the creation of {\tt Makefile}'s 
creates an output on the screen similar to:
%
{\small \begin{verbatim}
...
checking for mpicc option to accept ANSI C... none needed
checking for style of include used by make... GNU
checking dependency style of mpicc... gcc3
checking whether we are using the GNU C++ compiler... yes
checking whether mpiCC accepts -g... yes
checking dependency style of mpiCC... gcc3
checking for a BSD-compatible install... /usr/bin/install -c
checking for ranlib... ranlib
configure: creating ./config.status
config.status: creating Makefile
config.status: creating Src/Domain/Makefile
config.status: creating Src/Global/Makefile
config.status: creating Src/Scalar/Makefile
config.status: creating Src/Vector/Makefile
config.status: creating Src/Grid/Makefile
config.status: creating Src/Parallel/Makefile
config.status: creating Src/Parallel/Out/Makefile
...
\end{verbatim}}

\subsection{Compilation}

{\tt Makefile}'s for {\psiboil}, created in Sec.~\ref{sub_sec_creating_makefiles}, 
assume that the {\em main} program is in the file called {\tt main.cpp}, which resides 
in {\tt Src} sub-directory of the root directory. For convenience, we will call
it the {\em source} directory and, following examples from previous sections, it will 
be:
%
\begin{verbatim}
~/Development/PSI-Boil/Src
\end{verbatim}
%
The source directory is important because compilation is performed in it. Therefore,
you should now go to source directory. 

There is not {\tt main.cpp} (needed by {\tt Makefile}'s in source directory, so you 
should either create it, copy it from another location, or even create a link in source 
directory. The last option is recommended. So, assuming you are already in the source
directory, run the command:
%
\begin{verbatim}
> ln -i -s ../Doc/Tutorial/Volume1/Src/02-01-main.cpp main.cpp
\end{verbatim} 
%
When you have the desired main program in the source directory, you just run:
%
\begin{verbatim}
> make
\end{verbatim} 
%
and wait few minutes. {\psiboil} grew to be a complex and long program and
compilation requires patience. That is true only for the first build, as for
the sub-sequent, only the sources which have been modified are re-compiled,
thus greatly reducing the compilation time.

During the compilation, the screen will be filled with messages such as:
%
{\small \begin{verbatim}
Making all in Variable/Staggered/Momentum
make[1]: Entering directory `/home/niceno/Development/PSI-Boil/Src/Variable/Staggered/Mom
entum'
if mpiCC -DHAVE_CONFIG_H -I. -I. -I../../../..     -g -O2 -MT momentum.o -MD -MP -MF ".de
ps/momentum.Tpo" -c -o momentum.o momentum.cpp; \
then mv -f ".deps/momentum.Tpo" ".deps/momentum.Po"; else rm -f ".deps/momentum.Tpo"; exi
t 1; fi
if mpiCC -DHAVE_CONFIG_H -I. -I. -I../../../..     -g -O2 -MT momentum_bulk_bct.o -MD -MP
 -MF ".deps/momentum_bulk_bct.Tpo" -c -o momentum_bulk_bct.o momentum_bulk_bct.cpp; \
then mv -f ".deps/momentum_bulk_bct.Tpo" ".deps/momentum_bulk_bct.Po"; else rm -f ".deps/
momentum_bulk_bct.Tpo";exit 1; fi
if mpiCC -DHAVE_CONFIG_H -I. -I. -I../../../..     -g -O2 -MT momentum_bulk_ijk.o -MD -MP
 -MF ".deps/momentum_bulk_ijk.Tpo" -c -o momentum_bulk_ijk.o momentum_bulk_ijk.cpp; \
then mv -f ".deps/momentum_bulk_ijk.Tpo" ".deps/momentum_bulk_ijk.Po"; else rm -f ".deps/
momentum_bulk_ijk.Tpo";exit 1; fi
if mpiCC -DHAVE_CONFIG_H -I. -I. -I../../../..     -g -O2 -MT momentum_cfl_max.o -MD -MP 
-MF ".deps/momentum_cfl_max.Tpo" -c -o momentum_cfl_max.o momentum_cfl_max.cpp; \
then mv -f ".deps/momentum_cfl_max.Tpo" ".deps/momentum_cfl_max.Po"; else rm -f ".deps/mo
mentum_cfl_max.Tpo"; exit 1; fi
...
\end{verbatim}}
%
Once the compilation is done, the executable called {\tt Boil} will reside in the
source directory. 

\subsection{Running the code}

You can run the executable from the source directory by:
%
\begin{verbatim}
> ./Boil
\end{verbatim} 
%
Try it. It will do nothing. You should even try the parallel version:
%
\begin{verbatim}
> mpirun -np 2 ./Boil
\end{verbatim} 
%
It also does nothing. It is no wonder, since the main program has very little action
to it. It looks like:
%
{\small \begin{verbatim}
      1 #include "Include/psi-boil.h"
      2
      3 /****************************************************************************/
      4 main(int argc, char * argv[]) {
      5
      6   boil::timer.start();
      7
      8   boil::timer.stop();
      9 }
\end{verbatim}} 
%
The thing which immediately reveals this is different to ordinary {\tt C++} program 
is in line~1. This line includes definitions of all {\psiboil} objects. One of the
{\em global} {\psiboil}'s objects, {\tt timer}, is also used in this program\footnote
{It should be clear that a {\em global} {\psiboil} object does not mean global 
{\em class}, but an instance of object created when {\psiboil} starts, accessible 
from any part of the code.}.
It is stared in line~6 and stopped in line~8. The usage of {\tt timer} will be explained
below, with more elaborate examples. The program also reveals the {\em namespace}
defined with {\psiboil}, that is {\tt boil}. This namespace is used to enclose
all {\psiboil}'s global objects. 

% \clearpage

%---------------------------------------------------------------------nutshell-%
\vspace*{5mm} \fbox{ \begin{minipage}[c] {0.97\textwidth} %-----------nutshell-%
    {\sf Section \ref{sec_compiling} in a nutshell} \\ %--------------nutshell-%

    - The following software is required for compilation of {\psiboil}:
    \begin{itemize}
      \item a {\tt C++} compiler (Gnu's {\tt g++} will do the job),
      \item {\tt autotools},
      \item {\tt make} facility.
    \end{itemize}
    and is highle desirable if you have:
    \begin{itemize}
      \item {\tt MPI} libraries, as well.
    \end{itemize}
    On most Linux-based computers, all of the above software is available by default. \\

    - To compile the {\psiboil}, you have to perform the following steps:
    \begin{itemize}
      \item execute {\tt first LAM/MPICH} script in root directory 
            ({\tt .../PSI-Boil}),
      \item create {\tt main.cpp} in source directory ({\tt .../PSI-Boil/Src}),
            or link it to an already \\ existing program,
      \item run {\tt make} in {\em source} directory. 
    \end{itemize}
  \end{minipage} } %--------------------------------------------------nutshell-%
%---------------------------------------------------------------------nutshell-%
