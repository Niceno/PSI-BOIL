\label{chap_post_process}

This chapter describes several utilities for {\psiboil}.

\section{ChangeProc}
\label{sec_changeproc}
When you need to change the number of processes/cores used for the parallel computation at restart of computation, {\\t ChangeProc} can be used. 
\begin{verbatim}
	> cp Src/Utilities/ChangeProc
	> make
	> cp ChangeProc ~/.bin
\end{verbatim} 
The input files of {/tt ChangeProc} are {\tt "*.bck"} and the output files are {\tt "*.bck2"}. After you execute {\tt ChangeProc}, move the original backup files {\tt "*.bck"} to a backup folder, and change the file extension of {\tt bck2} to {\tt bck}.
\begin{verbatim}
	> tcsh
	> foreach i (*.bck2)
	  mv $i $i:r.bck
	  end
\end{verbatim} 

\section{ChangeGrid}
\label{sec_changeproc}
When you need to change the grid resolution at restart of computation, {\tt ChangeGrid} can be used. 
\begin{verbatim}
	> cp Src/Utilities/ChangeGrid
	> make
	> cp ChangeGrid ~/.bin
\end{verbatim} 
The file I/O of {\tt ChangeGrid} is as follows.
    \begin{itemize}
	\item Input: old and new grids, *.bck
	\item Output *.bck2
\end{itemize}
The old and new grid files must be prepared in advance.  In the old {\tt main.cpp}, include {\tt dom.save} as follows.
{\small \begin{verbatim}
  Domain dom(gx, gy, gz);
  dom.save("grid_old.grd");
\end{verbatim}}

Then, after you change the grid resolution, output the new grid.
{\small \begin{verbatim}
  Domain dom(gx_new, gy_new, gz_new);
  dom.save("grid_new.grd");
\end{verbatim}}
You run PSI-BOIL from the initial condition at this stage. Now, you have old and new grids and *.bck.  Then perform {\tt ChangeGrid} and obtain *.bck2. In the same was {\tt ChangeGrid}, change the file extension of bck2 to bck. Then you can restart the simulation using different grid resolution and different number of processes/cores.