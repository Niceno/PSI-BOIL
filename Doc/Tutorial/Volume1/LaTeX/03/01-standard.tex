\section{Standard terminal output}
\label{sec_standard}

In this section, we will develop a {\em classical} {\tt C/C++} application,
dating back to Karnaghan \& Ritchie famous book on {\tt C} programming 
language, {\em Hello World}. It is an application which does noting but
prints the message:
%
{\small \begin{verbatim}
Hello World!
\end{verbatim}}
%
on the terminal and yet it serves to illustrate some important features
of the language. In this tutorial, however, we will not use the standard
{\tt C++} language, but we will do it inside the {\psiboil} and we
will use it to illustrate {\psiboil}'s terminal output. We will 
use the program introduced in section~\ref{sec_compiling} as a starting
point. For this example, we introduce one more line which prints the desired 
message:
%
{\small \begin{verbatim}
      1 #include "Include/psi-boil.h"
      2
      3 /****************************************************************************/
      4 main(int argc, char * argv[]) {
      5
      6   boil::timer.start();
      7
      8   boil::aout << "Hello World!" << boil::endl;
      9
     10   boil::timer.stop();
     11 }
\end{verbatim}}
%
For your convenience, this program is already written, and to compile it,
you have to link it from the {\em source} directory ({\tt ...PSI-Boil/Src})
with the command:
%
\begin{verbatim}
> ln -i -s ../Doc/Tutorial/Volume1/Src/03-01-main.cpp main.cpp
\end{verbatim}
%
Once you make a link, remove the object file ({\tt main.o}) from the source
directory and re-make, {\em i.e.}:
%
\begin{verbatim}
> rm -f main.o
> make
\end{verbatim}
%
This will create the executable {\tt Boil} in the source directory. If you 
execute it, by issuing command:
%
\begin{verbatim}
> ./Boil
\end{verbatim}
%
from the source directory, you will get the following output:
%
{\small \begin{verbatim}
Hello World!
\end{verbatim}}
%
on the terminal. As you have probably guessed, the program line which creates 
this message is line number~8. In standard {\tt C++}, one would write:
%
{\small \begin{verbatim}
      8   std::cout << "Hello World!" << std::endl;
\end{verbatim}}
%
which is, of course, still possible in {\psiboil}, but is not recommended.
{\psiboil} has its own {\em global} objects for terminal output, suited
for parallel execution and execution on high-performance computing platforms. 
These objects are:
%
\begin{itemize}
  \item {\tt boil::aout} - acronym for {\tt a}ll {\tt out}, meaning that all 
                           processors print;
  \item {\tt boil::oout} - acronym for {\tt o}ne {\tt out}, meaning that only 
                           one processor prints;
  \item {\tt boil::endl} - {\psiboil}'s internal {\em end of line}. (This 
                           object was introduced to overcome very slow execution
                           of output on some high-performance computational platforms.
                           Namely, the {\tt C++} standard {\tt std::endl} was slowing
                           down the output on Cray-XT3 computer by two orders of 
                           magnitude.) 
  \item {\tt boil::pid} - processor i.d. for printing. (Not an integer value holding
                          the processor number). 
\end{itemize}
%

Now try to execute the application on four processors. Use the command:
%
\begin{verbatim}
> mpirun -np 4 ./Boil
\end{verbatim}
%
As you could expect, you get the following output:
%
{\small \begin{verbatim}
Hello World!
Hello World!
Hello World!
Hello World!
\end{verbatim}}
%
Each processor wrote its own message on the screen. You may wish to find out 
which processor wrote each of the messages. To find that out, you may use
the {\tt boil::pid}. Replace line~8 by:
%
{\small \begin{verbatim}
      8   boil::aout << boil::pid << "Hello World!" << boil::endl;
\end{verbatim}}
%
This program is already written and can be retrieved by:
%
\begin{verbatim}
> ln -i -s ../Doc/Tutorial/Volume1/Src/03-02-main.cpp main.cpp
\end{verbatim}
%
After linking, remove the {\tt main.o} and recompile the program with {\tt make}.
Re-run the program on four processors. You might get the output like this:
%
{\small \begin{verbatim}
0: Hello World!
1: Hello World!
3: Hello World!
2: Hello World!
\end{verbatim}}
%
This time, you get processor i.d.\ in front of each message. You may wonder,
at this point, what was the reason to introduce all these new objects just to
write messages on the screen, processors i.d.'s and end of lines 
({\tt boil::endl}). Well, the reason is {\em encapsulation} {\bf check}. 
{\psiboil} hides the details of implementation from the end-user (or developer). 
If {\tt MPI} standard is replaced with something else in the future, only the
{\psiboil} objects for terminal output will have to be changed (re-coded)
while for the rest of the code, which was using {\psiboil}'s global 
objects, nothing has to be changed.

Or, imagine a new hardware platform is developed in a number of years, for 
which the end of line ({\tt endl}) has to be defined in a different way for
efficiency. It will only have to be changed in one place, {\em i.e.} in the
definition of {\tt boil::endl}, while the rest of the code which uses that
global object to end the line can remain unchanged.

Let's assume now you want to print a different message. For example, if you
want to be sure you have reached the main body of the program, you would
want to write something like:
%
{\small \begin{verbatim}
      8   boil::aout << "Inside the PSI-Boil!" << boil::endl;
\end{verbatim}}
%
This program is in ({\tt ../Doc/Tutorial/Volume1/Src/03-03-main.cpp}). Re-link
it, recompile and run on one processor to get the message:
%
{\small \begin{verbatim}
Inside the PSI-Boil!
\end{verbatim}}
%
If you run it on eight processors, with:
%
\begin{verbatim}
> mpirun -np 8 ./Boil
\end{verbatim}
%
you will get:
%
{\small \begin{verbatim}
Inside the PSI-Boil!
Inside the PSI-Boil!
Inside the PSI-Boil!
Inside the PSI-Boil!
Inside the PSI-Boil!
Inside the PSI-Boil!
Inside the PSI-Boil!
Inside the PSI-Boil!
\end{verbatim}}
%
Clearly, this is an overkill. Some output should be written by one processor
only. To that end, the object {\tt boil::oout} should be used. If the line~8
is replaced with:
%
{\small \begin{verbatim}
      8   boil::oout << "Inside the PSI-Boil!" << boil::endl;
\end{verbatim}}
%
(it is in {\tt ../Doc/Tutorial/Volume1/Src/03-04-main.cpp}). Link to this
source, recompile and run. No matter how many processors you use, you will
always get one line of output:
%
{\small \begin{verbatim}
Inside the PSI-Boil!
\end{verbatim}}
%

%---------------------------------------------------------------------nutshell-%
\vspace*{5mm} \fbox{ \begin{minipage}[c] {0.97\textwidth} %-----------nutshell-%
    {\sf Section \ref{sec_standard} in a nutshell} \\ %---------------nutshell-%

    - {\psiboil}'s terminal output should be performed using the objects:
    %
    \begin{itemize}
      \item {\tt boil::aout} - if all processors print,
      \item {\tt boil::oout} - if one processor prints,
      \item {\tt boil::endl} - to end the line,
      \item {\tt boil::pid}  - for printing processor's i.d.
    \end{itemize}
    
    - The usage of standard {\tt C++} output facilities, such as {\tt std::cout},
    {\tt std::endl}, as well as {\tt MPI}'s standard commands for getting 
    processor i.d.\ is {\em discouraged}. 
  \end{minipage} } %--------------------------------------------------nutshell-%
%---------------------------------------------------------------------nutshell-%

