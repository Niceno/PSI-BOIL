\section{Global timing}
\label{sec_global}

The examples considered so far executed almost instantly and it does not
make sense to measure their performance. Here is a program example
which has a cognizable CPU-time of execution ({\tt 04-01-main.cpp})\footnote{From
this point on, the example program names will be given without their location,
because it is assumed they all reside in directory: {\tt PSI-Boil/Doc/Tutorial/Volume1/Src}.
Furthermore, the procedures for linking them into source directory and compiling
them will not be repeated.}:
%
{\small \begin{verbatim}
      1 #include "Include/psi-boil.h"
      2
      3 const int N = 256;
      4
      5 /****************************************************************************/
      6 main(int argc, char * argv[]) {
      7
      8   boil::timer.start();
      9
     10   real a;
     11
     12   boil::oout << "Running ..." << boil::endl;
     13
     14   for(int i=0; i<N; i++)
     15     for(int j=0; j<N; j++)
     16       for(int k=0; k<N; k++) {
     17         a = acos(3.14);
     18       }
     29
     20   boil::timer.stop();
     21   boil::timer.report();
     22 }
\end{verbatim}}
%
The central part of this program is nested loop in lines~14--16 which computes
$acos(3.14)$. Clearly, this program will calculate $acos$ (a relatively 
time-consuming operation) for $256^3$ times. The commands {\tt boil::timer.start();}
and {\tt boil::timer.stop();} (lines 8 and 20 respectively) start and stop the 
global timer. In addition to these, we also use {\tt boil::timer.report();}, which
will report (print on the terminal) the time needed to execute this program. 
Compile and run this program, to get an output such as this\footnote{The actual 
time in seconds will differ on your computer, of course}: 
%
{\small \begin{verbatim}
Running ...
+==========================================
| Total execution time: 5.14 [s]
+------------------------------------------
\end{verbatim}}
%
As you can see {\psiboil} reports on the total CPU-time needed for program 
execution. You only had to use commands:
%
\begin{itemize}
  \item {\tt boil::timer.start();} - as the first,
  \item {\tt boil::timer.stop();} - as penultimate, and
  \item {\tt boil::timer.report();} - as the last statement in the program.
\end{itemize}
%
If you call {\tt boil::timer.stop();} before {\tt boil::timer.start();}, or
{\tt boil::timer.report();} before any of the other two, error messages will
be printed on the terminal.

Although the main topic of this section is timing of {\psiboil} programs,
two more issues deserve attention at this point:
%
\begin{itemize}
  \item Line 10: {\tt real a;} - declaration of floating point number which
        is not one of the standard {\tt C++} data types (neither {\tt float}
        nor {\tt double}). It is essentially a macro, defined in:
        {\tt Global/global\_precision.h} as one of the standard floating
        point types. The reason for introducing it, is to make precision of
        {\psiboil} independent of compiler options, which may be tedious
        to remember for different compilers or platforms. The usage of {\tt MPI}
        libraries (which have their own definitions of single and double precision
        numbers) makes things even less transparent. With macro for {\tt real},
        things are quite straightforward: if you want the entire {\psiboil} 
        to run in single precision, define {\tt real} as {\tt float},
        if you want it in double precision, just define {\tt real} as {\tt double}.
        Use of standard declarations is {\em strongly discouraged}, since it would
        lead to inconsistencies between different parts of the program.
  \item Line 3 defines an integer constant which sets the limits for loops in 
        lines~14--16. That is a good programming practice. If you kept number~256
        in the loops itself, every time you want to change the problem size, you would
        have to change all three lines, not to mention that in a bigger program
        you might forget what all these numbers mean. Such {\em ghost} numbers
        greatly reduce the modifiability of the code and should be avoided by any
        means.
\end{itemize}

%---------------------------------------------------------------------nutshell-%
\vspace*{5mm} \fbox{ \begin{minipage}[c] {0.97\textwidth} %-----------nutshell-%
    {\sf Section \ref{sec_global} in a nutshell} \\ %-----------------nutshell-%
    
    - To measure the total CPU-time used by {\psiboil}
    use commands: 
    \begin{itemize}
      \item {\tt boil::timer.start();} - as the first,
      \item {\tt boil::timer.stop();}  - as the penultimate, and
      \item {\tt boil::timer.report();} - as the last statement in the 
                                          program.
    \end{itemize}

    - Floating point numbers are defined by macro {\tt real}, which is defined
    to be on of the standard {\tt C++} data types {\tt float} or 
    {\tt double}. \\ 

    - Use of {\em ghost} numbers should be avoided. 

  \end{minipage} } %--------------------------------------------------nutshell-%
%---------------------------------------------------------------------nutshell-%
