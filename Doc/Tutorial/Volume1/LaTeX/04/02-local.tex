\section{Local timing}
\label{sec_local}

In the section~\ref{sec_global} we measured the total time needed
for program execution. Imagine you want to compare two algorithms,
one using the {\em acos} function, and the other using {\em cos}
(04-02-main.cpp)\footnote{These algorithms obviously serve no real 
purpose but are used solely to illustrate timing features}:
%
{\small \begin{verbatim}
      1 #include "Include/psi-boil.h"
      2
      3 const int N = 256;
      4
      5 /****************************************************************************/
      6 main(int argc, char * argv[]) {
      7
      8   boil::timer.start();
      9
     10   real a;
     11   real b;
     12
     13   boil::oout << "Running acos ..." << boil::endl;
     14
     15   /* algorithm using "acos" */
     16   for(int i=0; i<N; i++)
     17     for(int j=0; j<N; j++)
     18       for(int k=0; k<N; k++) {
     19         a = acos(-1.0));
     20       }
     21
     22   boil::oout << "Running cos ..." << boil::endl;
     23
     24   /* algorithm using "cos" */
     25   for(int i=0; i<N; i++)
     26     for(int j=0; j<N; j++)
     27       for(int k=0; k<N; k++) {
     28         a = cos(-1.0);
     29       }
     30
     31   boil::timer.stop();
     32   boil::timer.report();
     33 }
\end{verbatim}}
%
If you compile and run it, you will get the following output:
%
{\small \begin{verbatim}
Running acos ...
Running cos ...
+==========================================
| Total execution time: 7.21 [s]
+------------------------------------------
\end{verbatim}}
%
The execution time is measured as above, but the information on the 
total time is not quite useful now. You would like to see how much
time each of the algorithms ({\em acos} and {\em cos}) consumes.
Object {\tt boil::timer} can be used for that purpose too. To achieve
that, you must enclose the parts of the code you would like to 
measure with {\tt boil::timer.start()} and {\tt boil::timer.stop()},
but with the {\em names} of the parts of the algorithm sent as
parameters. To illustrate it, lines~15--29, should be replaced by the 
following lines~({\tt 04-03-main.cpp}):
%
{\small \begin{verbatim}
     15   /* algorithm using "acos" */
     16   boil::timer.start("acos algorithm");
     17   for(int i=0; i<N; i++)
     18     for(int j=0; j<N; j++)
     19       for(int k=0; k<N; k++) {
     20         a = acos(-1.0);
     21       }
     22   boil::timer.stop("acos algorithm");
     23
     24   boil::oout << "Running cos ..." << boil::endl;
     25
     26   /* algorithm using "cos" */
     27   boil::timer.start("cos algorithm");
     28   for(int i=0; i<N; i++)
     29     for(int j=0; j<N; j++)
     30       for(int k=0; k<N; k++) {
     31         a = cos(-1.0);
     32       }
     33   boil::timer.stop("cos algorithm");
\end{verbatim}}
%
The lines which are different to previous program ({\tt 04-02-main.cpp}) are
in lines: 16, 22, 27 and 33. Lines 16 and 22 enclose the {\em acos} algorithm
and send the name {\tt "acos algorithm"} as parameter, while the lines 27
and 33 enclose the {\em cos} algorithm and send the name {\tt "cos algorithm"}
as parameter. These names are arbitrary, but must be {\em unique}. Furthermore,
each {\tt boil::timer.start(char * name);} must have a corresponding 
{\tt boil::timer.stop(char * name);}. If you compile and run this program, you will
get the following output:
%
{\small \begin{verbatim}
Running acos ...
Running cos ...
+==========================================
| Total execution time: 7.05 [s]
+------------------------------------------
| Time spent in acos algorithm: 5.91 [s]    (83.8298%)
| Time spent in cos algorithm : 1.14 [s]    (16.1702%)
| Time spent elsewhere: 0 [s]    (0%)
+------------------------------------------
\end{verbatim}}
%
This output quite apparently shows how much time was spent in each part of
the code that you wanted to measure. The names you have assigned in calls
to {\tt boil::timer.start("name")} and {\tt boil::timer.stop("name");}
({\tt "acos algorithm"} and {\tt "cos algorithm"}) appear in the output,
showing how much time was spend in each of them and what percentage of
total CPU-time they consume. This information is a clear indication 
to parts of the code which need optimization for speed. In other words, 
{\psiboil} has built-in {\em profiling} capability. All these features work
for parallel version as well. The information which parallel version would
print on the terminal is the average it spent over all processors. If you
wanted to have separate information for each processor, you should
use {\tt boil::timer.report\_separate();} instead of 
{\tt boil::timer.report();}.

It was mentioned above that each call to {\tt boil::timer.start(char * name);} 
requires a call to {\tt boil::timer.stop(char * name);}. It does not have to
mean that number of {\tt boil::timer.start(char * name);} is always equal
to number of {\tt boil::timer.stop(char * name);} in a program unit. Imagine
you are timing a subroutine which has two possible exits, typical for iterative 
algorithms\footnote{This is just an illustrative example, not a real program
which could be compiled inside the {\psiboil}. It's source is therefore not 
available in a separate file}:
%
{\small \begin{verbatim}
      1 void iterative(const int n, const real res) {
      2
      3   boil::timer.start("iterative");
      4
      5   /* start the iterative procedure */
      6   for(int it=0; it<n; it++) {
      7      ...
      8      ...
      9      ...
     10      if(residual < res) {        
     11        boil::oout << "algorithm converged!" << boil::endl;
     12        boil::timer.stop("iterative");
     13        return;
     14      }
     15   }
     16
     17   boil::oout << "algorithm failed to converge!" << boil::end;
     18   boil::timer.stop("iterative");
     19   return;
     20 }
\end{verbatim}}
%
In this example, the whole subroutine {\tt iterative(const int, const real)}
is measured. Timing starts in line~3, but it can end in two ways. If the desired
criteria is met inside the iterative procedure (lines~6--15), you exit from the 
subroutine in line~13. If the iterative procedure fails to meet the desired
criteria, you exit from the subroutine in line~20. A call to 
{\tt boil::timer.stop(char * name);} must be present for each of these cases.
In the above example it is. In lines~12 and~18. 

%---------------------------------------------------------------------nutshell-%
\vspace*{5mm} \fbox{ \begin{minipage}[c] {0.97\textwidth} %-----------nutshell-%
    {\sf Section \ref{sec_local} in a nutshell} \\ %------------------nutshell-%

      - To measure the CPU-time spent in one part of the code, enclose 
      this part with commands:
      \begin{itemize}
        \item {\tt boil::timer.start(char * name);} 
        \item {\tt boil::timer.stop(char * name);} 
      \end{itemize}
      where {\tt name} is a {\em unique} name, and it will appear in the 
      output of the final call to \\ {\tt boil::timer.report()}. \\

      - Each call to {\tt boil::timer.start(char * name);} {\em must} have
      a corresponding call to {\tt boil::timer.stop(char * name);}. 

  \end{minipage} } %--------------------------------------------------nutshell-%
%---------------------------------------------------------------------nutshell-%
