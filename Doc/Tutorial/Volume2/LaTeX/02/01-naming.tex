\section{Naming conventions}
\label{sec:naming_conventions}

%%%%%%%%%%%%%%%%%
%               %
%  Class names  %
%               %
%%%%%%%%%%%%%%%%%
\subsection{Class names}

In {\psiboil}, classes are named with first letter capitalized, such as:
\begin{itemize}
  \item {\tt Scalar}, 
  \item {\tt Vector}, 
  \item {\tt Domain}, etc. 
\end{itemize}

If the description of a class requires two words, they are written together,
with each word having the first letter capitalized. Examples include:
\begin{itemize}
  \item {\tt LevelSet} and 
  \item {\tt RandomFlow}. 
\end{itemize}

In cases where a class with best described with an acronym, the name of the
class is capitalized, such as:
\begin{itemize}
  \item {\tt CIPCSL2} for Constrained Interpolated Profile method of Conservative 
        Semi-Lagrangian of the 2$^{nd}$ order.
  \item {\tt AC} which stands for Additive Correction.
\end{itemize}

Combinations of words and acronyms are also allowed:
\begin{itemize}
  \item {\tt Grid1D} for Grid in 1 Dimension, or             
  \item {\tt ColorCIP} for Color Cubical Interpolated Polynomial.
\end{itemize}

This naming convention makes the {\psiboil}'s classes (which may be regarded as
new data types), distinguishable from {\tt C++}'s built-in data types.
%
The naming convention explain here holds for {\em class} names. 
The {\em objects} derived from these classes are named as other variables, 
generally in low-case only.

%%%%%%%%%%%%%%%%%%%%%%%%%%%%%%%%%
%                               %
%  Function and variable names  %
%                               %
%%%%%%%%%%%%%%%%%%%%%%%%%%%%%%%%%
\subsection{Function and variable names}

All variables, no matter if they are of {\tt C++}'s built-in data type, or of
{\psiboil} class, are written in lower case. If a name is made of more than
one word, it is separated with an underscore. 

Some examples of variable declarations are:
{\small \begin{verbatim}
  int i;                     /* built-in data type */  
  real x_i;                  /* built-in data type */  
  Body body("cylinder.stl"); /* a PSI-Boil object */ 
  Domain d(gx,gy,gz);        /* a PSI-Boil object */ 
\end{verbatim}}

In some instances, this rule can be violated. For example, if a complex 
algorithm is implemented, which follows a publication where small and capital
letters are used simultaneously, keeping the same notation in the code makes
the algorithms easier to follow. Example are the interpolation and restriction
part of the AC algorithm, whose implementation is in the functions 
{\tt additive\_interpolation.cpp} and {\tt additive\_restriction}, which 
begin with:
%
{\small \begin{verbatim}
     12 void AC::interpolation(const Centered & H, Centered & h) const {
\end{verbatim}}
%
and:
%
{\small \begin{verbatim}
     12 void AC::restriction(const Centered & h, Centered & H) const {
\end{verbatim}}
%
respectively. Here, the coarse and the fine variable are denoted with
an {\tt H} and {\tt h} respectively, the notation introduced in the
original paper. 

Such exceptions should be kept at the minimum and {\em only} applied locally,
i.e.\ inside function definitions, for local variables. A global variable, or
a variable belonging to a class, must be written in small case.

%%%%%%%%%%%%%%%%%%%%
%                  %
%  Constant names  %
%                  %
%%%%%%%%%%%%%%%%%%%%
\subsection{Constant names}

There are two simple ways two define constants in {\tt C++}:

\begin{itemize}
  \item Using {\tt C/C++} directive {\tt \#define} -- This directive is inherited
        from the {\tt C} language. It is insecure since it is based on a plain
        text replacement and thus no type checking is performed by compiler. 
        This method is {\bf obsolete} and should be {\bf avoided}.
  \item Using {\tt C++}'s declaration {\tt const} -- This is a modern way to define
        a constant in {\tt C++} program. It is much safer than the {\tt \#define}
        directive, since the {\tt const} declaration is merely a modifier for
        a data type which follows. This method is strongly promoted in {\psiboil}
        throughout the code.
\end{itemize}

Each of these types of constants follows different naming conventions. Constant 
defined by {\tt \#define} directive (obsolete) are defined with all upper-case
letters. This is done for two reasons. One is historical. This directive is
inherited from {\tt C}, where constants were written with all capital letters.
The other reason to keep them in upper case is for easier identification. Since
{\psiboil} is aimed to be a well-designed program following state-of-the art
advances in software development, such constants should be eradicated. Actually,
at the time of writing this manual, only one such constant is present in the
code\footnote{Here we mean usage of {\tt \#define} for constants. It is still
heavily for defining certain macro's, as well as for compiler directives.}. 
It is in file {\tt communicator.h}:
%
{\small \begin{verbatim}
      4 #define MAX_BUFF_SIZE 1048576
\end{verbatim}}
%
Constants defined using the {\tt C++}'s {\tt const} declaration, are named with
the same convention as variables. Hence, they are written in lower-case. There 
are two reasons for that. First is to make them different from old-style constant
declaration. The other lies in the fact that {\tt const} directive is heavily
used inside {\psiboil} in all parts of the code: from constants local to certain
sections of code to parameters. For example, in the same {\tt communicator.h} 
file introduced above, we have:
%
{\small \begin{verbatim}
     58     void send(real * buff_send, const int size,
     59               const par_datatype & par_t,
     60               const int dir_send,
     61               const Tag tag) const;
\end{verbatim}}
%
all but the first parameter are constants. Typing them all in upper-case would
make the code far less readable.

%%%%%%%%%%%%%%%%%%%%%%%%%%%%%%%%%
%                               %
%  Reference and pointer names  %
%                               %
%%%%%%%%%%%%%%%%%%%%%%%%%%%%%%%%%
\subsection{Reference and pointer names}

Names for references and pointers follow the same rule as variables.
However, for readability, one space must be placed between the
reference or pointer sign ({\tt \&} or {\tt *}), and the variable name.
Line~86 from file {\tt body.h} illustrates this rule:
%
{\small \begin{verbatim}
    86     void ijk(Comp & m, int cc, int * i, int * j, int * k) const;
\end{verbatim}}
%
A {\bf wrong} way would be to attach reference and pointer signs to variable
names: 
%
{\small \begin{verbatim}
    86     void ijk(Comp &m, int cc, int *i, int *j, int *k) const;
\end{verbatim}}
%

%%%%%%%%%%%%%%%%%%%%%%%%%%%%%%%%%%%%%
%                                   %
%  Directory and source file names  %
%                                   %
%%%%%%%%%%%%%%%%%%%%%%%%%%%%%%%%%%%%%
\clearpage
\subsection{Directory and source file names}

\subsubsection{Directory names}

Directory names follow the same convention as classes themselves: they are
written with first letter capitalized. This convention follows the principle
that each class (except for very small, {\em ravioli} type classes is in a
separate directory. 

As an example, here is a part of {\psiboil}'s directory structure:
%
{\small \begin{verbatim}
|-- Equation
|   |-- Centered
|   |   |-- ColorCIP
|   |   |-- Concentration
|   |   |-- Distance
|   |   |-- Enthalpy
|   |   |-- LevelSet
|   |   `-- Pressure
|   `-- Staggered
|       `-- Momentum
\end{verbatim}}
%
In the {\tt Equation} directory, the class {\tt Equation} is defined, with it's
children: {\tt Centered} and {\tt Staggered} in it's sub-directories with same 
names. {\tt Centered} has six children, each defined in it's sub-directory,    
named the same as the class it contains.

\subsubsection{Source file names}

To define a class, one needs a header file ({\tt .h}) and one or more
source files ({\tt .cpp}). The files which define a certain class, 
reside in it's directory. The file names are written in low-case letters only,
the file name beginning with class name, followed by the function name. 

To illustrate this point with an example, bellow is the list of source files to
define object {\tt LevelSet} 
(residing in directory {\tt Src/Equation/Centered/LevelSet}):
%
{\small \begin{verbatim}
CVS
levelset.cpp
levelset_discretize.cpp
levelset.h
levelset_sharpen.cpp
levelset_tension.cpp
Makefile.am
\end{verbatim}}
%
In this example, the header file {\tt levelset.h} contains the declaration of 
the object {\tt LevelSet}, whereas {\tt levelset.cpp} contains constructor and 
destructor functions. Small member function (like setter and getter) are defined 
in the header file. Bigger functions, such as {\tt LevelSet::discretize}, 
{\tt LevelSet::discretize} and {\tt LevelSet::tension} are defined in separate
source files. 

The files defining a class are placed in a directory with class' name, so 
inserting the class name at the beginning of each file's name may seem like an
overkill. But, it proves to be very useful when one prints source files. Since
there may be more than one class with a member function with a same name, one
might mix up their paper copies. If a class name precedes the file name, it
will not happen. 

%---------------------------------------------------------------------nutshell-%
\vspace*{5mm} \fbox{ \begin{minipage}[c] {0.97\textwidth} %-----------nutshell-%
    {\sf Section \ref{sec:naming_conventions} in a nutshell} \\ %-----nutshell-%

      - Class names in {\psiboil}:
      \begin{itemize}
        \item have first letter capitalized ({\tt Scalar}),
        \item if created from more than one word, each word has first letter 
              capitalized ({\tt LevelSet}),
        \item if based on an acronym, all letters are capitalized ({\tt AC}),
        \item can be defined as combinations of the above ({\tt ColorCIP}).
      \end{itemize}
      
      - Function and variable names in {\psiboil}:
      \begin{itemize}
        \item are written with all letters in lower-case,
        \item if created from more than one word, words are separated by 
              an underscore character.
      \end{itemize}
      
      - Reference and pointer signs must not be attached to variable names. \\
      
      - Constant names in {\psiboil}:
      \begin{itemize}
        \item are named with all letters in upper-case if defined with
              the obsolete {\tt C/C++} directive {\tt \#define}.
        \item follow the same convention as variables, if defined with
              the {\tt C++}'s {\tt const} declaration.
      \end{itemize}
      
      - Directory names in {\psiboil}:
      \begin{itemize}
        \item follow the same rules as class names.     
        \item Directory hierarchy, closely follows class hierarchy.      
              Inherited classes are placed in parent's sub-directories.
      \end{itemize}
      
      - File names in {\psiboil}:
      \begin{itemize}
        \item are written in lower-case letters only,   
        \item if formed from more than one word, they are separated by
              underscores,
        \item if belonging to a class (most likely) the first word in the
              file name must be the class name in lower-case.
      \end{itemize}
  \end{minipage} } %--------------------------------------------------nutshell-%
%---------------------------------------------------------------------nutshell-%
