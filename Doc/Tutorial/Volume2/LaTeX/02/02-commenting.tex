\clearpage
\section{Commenting the source code}
\label{sec:commenting}

There are two types of comments in {\psiboil}:
\begin{itemize}
  \item {\psiboil} style comments
  \item Doxygen style comments
\end{itemize}

\subsection{{\psiboil}-style comments}

\subsubsection{Class definition}

In front of each class definition, a special frame-like comment is inserted, which
makes the beginning of class definition more visible. For example, comment 
preceding the definition of class {\tt Scalar}, taken from the file {\tt scalar.h}
reads:
%
{\small \begin{verbatim}
     17
     18 //////////////
     19 //          //
     20 //  Scalar  //
     21 //          //
     22 //////////////
     23 class Scalar {
     24   public:
     25     /* global constructor */
            ...
\end{verbatim}}
% 
The frame-like comment preceding the class is in lines~18--22. It is five lines
long. First and last line contain only forward slashes, third line contains the
exact class name preceded by double forward slash and double space. Class name
is followed by double space and two more forward slashes. The width of this frame
is thus equal to the number of letters in class name, plus four.

Frame-like comments such as:
%
{\small \begin{verbatim}
     17
     18 //////////////////
     19 //              //
     20 //  Scalar  //
     21 //              //
     22 //////////////////
     23 class Scalar {
     24   public:
     25     /* global constructor */
            ...
\end{verbatim}}
% 
or:
%
{\small \begin{verbatim}
     17
     18 //////////
     19 //     //
     20 //  Scalar  //
     21 //      //
     22 //////////
     23 class Scalar {
     24   public:
     25     /* global constructor */
            ...
\end{verbatim}}
% 
are {\bf not} allowed.

\subsubsection{Function definition}

Each function definition should be preceded by a single-line, separator-like
comment such as the line~29 in the following fragment of the code:
%
{\small \begin{verbatim}
     29 /******************************************************************************/
     30 Scalar::Scalar(const Scalar * s) : alias(true) {
     31 /*------------------------------------+
     32 |  this constructor creates an alias  |
     33 +------------------------------------*/
     34 
\end{verbatim}}
% 
The length of the separator line is exactly 80 characters. It starts and ends 
with a forward slash, in-between which there are 78 stars ({\tt *}). 
No other characters than starts ({\tt *}) are allowed here.

If a function definition requires an explanation, it should be framed as shown
in lines~31--33 in the above example. The frame starts with a {\tt /*} and ends
 with a {\tt */} combination. Horizontal frame lines are created from 
minuses~({\tt -}) and vertical lines from {\tt |}'s. The upper right and bottom 
left corner of this frame are created from a plus sign~({\tt +}). 

The width of the frame is variable, adjusted to text width. There are two spaces
between the text and vertical lines ({\tt |}), which makes the frame width
equal to the text width plus six. 

Few separators which are {\bf not allowed} in {\psiboil} are:
%
{\small \begin{verbatim}
     44 /*############################################################################*/
         
     63 /*----------------------------------------------------------------------------*/
        
     88 /*%%%%%%%%%%%%%%%%%%%%%%%%%%%%%%%%%%%%%%%%%%%%%%%%%%%%%%%%%%%%%%%%%%%%%%%%%%%%*/
        
    132 /****************************************************************************/
\end{verbatim}}
% 
The  first three are built from characters other than {\tt *}, the last one is too
short.

Frame examples which are {\bf not allowed} in {\psiboil} follow:
%
{\small \begin{verbatim}
     23 /*-----------------------+
     24 |this frame is too narrow|
     25 +-----------------------*/
        ...
     67 /*-------------------------------+
     68 |     this frame is too wide     |
     69 +-------------------------------*/
        ...
    107 /*!!!!!!!!!!!!!!!!!!!!!!!!!!!!!!!!!!!!!!!!
    108 !  the width of this frame is good, but  !
    109 !   it is built from wrong characters    !
    110 !!!!!!!!!!!!!!!!!!!!!!!!!!!!!!!!!!!!!!!!*/
\end{verbatim}}
% 

\subsubsection{Code segments}

Very often, code segments have to be commented as well. It is also clear that
these comments may have different significance. Some may note beginning of a
very important loop, the others some less important fractions of the code, 
others still might be just insignificant reminders. These different levels
of comment importance, are illustrated in the pseudo-code below:
%
{\small \begin{verbatim}
     24 /*---------------------------------------+
     25 |                                        |
     26 |  a very important comment is created   |
     27 |  using the rules for framing function  | 
     28 |  descriptions, but with an additional  |
     29 |  blank line on the top and the bottom  |
     30 |                                        |
     31 +---------------------------------------*/
        
     88   /*--------------------------------------+
     89   |  a less important comment is created  |
     90   |   using exactly the same rules for    |
     91   |     framing function descritions.     |
     92   +--------------------------------------*/
  
    107     /* even less important single line comment */

    113     /* the same importance as above
    114        but taking more than one line */

    150       int c; /* counter (insignificant reminder) */
\end{verbatim}}
 
All {\psiboil}-style function comments are written using lower-case letters only, 
thus violating grammatical rules. That makes easily distinguishable from 
Doxygen-style comments introduced bellow.

Single-line comments in {\tt C++} style (starting with double slash {\tt //}) are
{\bf not} allowed for comments describing the code. They are allowed only for
debugging statements, so that a developer can activate them rapidly. For example,
the file {\tt pressure\_update\_rhs.cpp} contains the lines:
%
{\small \begin{verbatim}
     99   //debug: if( time->current_step() % 100 == 0 ) 
    100   //debug:   boil::plot->plot(fnew, "p-fnew", time->current_step());
          ...
\end{verbatim}}
%
which may be activated in cases when solution to the pressure equation 
diverges. But, these comments enclose debugging statements. Single-line
{\tt C++} style comments should not be used to comment code functionality.
Hence:
%
{\small \begin{verbatim}
              ...
    150       int c; // counter (insignificant reminder)   
\end{verbatim}}
%
is {\bf not} allowed.

\subsection{Doxygen-style comments}

As the name clearly implies, Doxygen-style comments are used to allow creation
of code documentation using the Doxygen package. Doxygen allows a lot of 
flexibility in writing comments, but in {\psiboil} a set of rules is set up,
as described bellow. At the time of creating this document, Doxygen-style
comments are inserted mostly into {\em header} files.

At the beginning of each class definition, one should use the following style
of the comment (taken from the file {\tt bndcnd.h}):
%
{\small \begin{verbatim}
     67 /***************************************************************************//**
     68 *  \brief Ravioli class for safer specification of arguments representing
     69 *         boundary condition values.
     70 *
     71 *  It may hold either a real value if boundary condition is specified with a 
     72 *  single constant, or with a string (character array), if boundary condition
     73 *  is specified with an analytical expression.
     74 *******************************************************************************/
\end{verbatim}}
% 
The width of the comment is exactly 80 characters. Such a comment contains two
distinguished parts:
%
\begin{itemize}
  \item Brief description -- A sentence which follows Doxygen's command 
        {\tt brief}. In the above example, the brief description is in the 
        lines~68 and~69. Brief description appears in Doxygen's list of classes.
  \item Broader description -- Anything following brief description till the end
        of the comment. This description appears when one clicks on the class
        name and can be as detailed as one wishes. It can contain equations, 
        pictures and code samples.
\end{itemize}
%
Inside the body of a class definition, brief single-line comments can be used.
An example from the file {\tt monitor.h} illustrates it:
%
{\small \begin{verbatim}
     20 class Monitor {
     21   public:
     22     //! Prints the value of Scalar at specified position(s).
     23     virtual void print(const Scalar & u) = 0;
        ...
\end{verbatim}}
% 
In addition to these, single-line brief descriptions, list of parameters can 
also be specified, using Doxygen's command {\tt param} as illustrated with the
following code:
%
{\small \begin{verbatim}
            ...
     25     //! Prints the value of a Vector component at specified position(s).
     26     /*! 
     27         \param u - Vector to be printed,
     28         \param m - specifies the Vector component.
     29     */
     30     virtual void print(const Vector & u, const Comp & m) = 0;
\end{verbatim}}
% 
also taken from {\tt monitor.h}.

All {Doxygen}-style comments are written as real sentences, using as correct
grammatical rules as possible. That shows they will be used for documentation
and makes them distinguishable from {\em Spartanic}, {\psiboil}-style comments 
introduced above.

%---------------------------------------------------------------------nutshell-%
\vspace*{5mm} \fbox{ \begin{minipage}[c] {0.97\textwidth} %-----------nutshell-%
    {\sf Section \ref{sec:commenting} in a nutshell} \\ %-------------nutshell-%

      - Two different styles of comments are allowed and used in the code:
      \begin{itemize}
        \item Doxygen-style comments,
        \item {\psiboil}-style comments.
        \item For each style a number of rules are defined.
        \item Doxygen-style comments are written in grammatically correct
              English.
        \item {\psiboil}-style comments are written with lower-case letters only.
      \end{itemize}
 
      - Independently of the commenting style:
      \begin{itemize}
        \item no comment should be wider than 80 characters     
        \item commenting rules should be obeyed by all {\psiboil} developers.
      \end{itemize}
  \end{minipage} } %--------------------------------------------------nutshell-%
%---------------------------------------------------------------------nutshell-%
