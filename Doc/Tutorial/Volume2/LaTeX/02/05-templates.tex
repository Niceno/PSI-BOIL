\section{Template header and source files}
\label{sec:templates}

This chapter is wrapped up by providing template header and source files.
The files provided here are a part of {\psiboil} sources and reside in
directory {\tt Src/Template}. The sources define the class {\tt Template}. 

Class {\tt Template} serves primarily a demonstration purpose; it illustrates
the points outlined in this chapter. It could be useful as a starting point to
adding new classes into {\psiboil}. It's only real functions is to store two
real data members, which can be accessed with setter and getter functions, and
it can also return the sum of their squares.

The entire class is declared in file {\tt template.h}, included bellow:
%
{\small \begin{verbatim}
      1 /***************************************************************************//**
      2 *  \brief A template header file for PSI-Boil
      3 *
      4 *  This file was created as a part of the PSI-Boil Tutorial, Volume 2:
      5 *  Development Manual. It's purpose is to outline components a PSI-Boil
      6 *  header file should have. It can be used as a starting point when defining
      7 *  new classes.
      8 *
      9 *  \warning
     10 *  Although it is compiled with the package, it has no functional purposes.
     11 *******************************************************************************/
     12
     13 /* directives for the compiler */
     14 #ifndef TEMPLATE_H
     15 #define TEMPLATE_H
     16
     17 #include <iostream>                     /* first stadard C++ headers  ...     */
     18 #include "../Global/global_precision.h" /* ... followed by PSI-Boil's headers */
     19
     20 ////////////////
     21 //            //
     22 //  Template  //
     23 //            //
     24 ////////////////
     25 class Template {
     26   public:
     27     //! Default construtor.
     28     Template() {at=0; bt=0;}  /* if short, it can stay in the header */
     29
     30     //! A more elaborate constructor.
     31     Template(const real & x,
     32              const real & y); /* if long, defined in a .cpp source file */
     33
     34     ~Template();
     35
     36     //! Getter and setter functions should be in the header
     37     real a() const {return at;}    /* getter should be declared as const */
     38     real b() const {return bt;}    /* getter should be declared as const */
     39     void a(const real & x) {at=x;} /* const promoted for parameters */
     40     void b(const real & y) {bt=y;} /* const promoted for parameters */
     41
     42     //! Returns the sum of squares.
     43     real sum_squares() const;
     44
     45   private:
     46     real at, bt;
     47 };
     48
     49 #endif /* directive for the compiler */
     50
     51 /*-----------------------------------------------------------------------------+
     52  '$Id: 05-templates.tex,v 1.1 2009/11/12 15:29:43 niceno Exp $'/
     53 +-----------------------------------------------------------------------------*/
\end{verbatim}}
% 
In addition to declaring the entire object, the header file also defines default 
constructor ({\tt Template}), setter and getter functions (variants of {\tt a} and 
{\tt b}). The non-default constructor, supposed to be complex, for the sake of the
argument, and is therefore defined in a separate file {\tt template.cpp}.
%
{\small \begin{verbatim}
      1 #include "template.h" /* class header must be included */
      2
      3 /***************************************************************************//**
      4 *  This is a non-trivial constructor. Note that this comment is written in
      5 *  Doxygen style and therefore in a grammatically correct English language.
      6 *******************************************************************************/
      7 Template :: Template(const real & x, const real & y) {
      8
      9   /*----------------------------------------------+
     10   |  copy values of paremeters into data members  |
     11   +----------------------------------------------*/
     12   at = x;
     13   bt = y;
     14 }
     15
     16 /******************************************************************************/
     17 Template :: ~Template() {
     18 /*--------------------------------------------------------------------------+
     19 |  destructor is also placed here.                                          |
     20 |  this is a psi-boil-style comment, written with lower-case letters only.  |
     21 +--------------------------------------------------------------------------*/
     22
     23 }
     24
     25 /*-----------------------------------------------------------------------------+
     26  '$Id: 05-templates.tex,v 1.1 2009/11/12 15:29:43 niceno Exp $'/
     27 +-----------------------------------------------------------------------------*/
\end{verbatim}}
% 
The file {\tt template.cpp}, i.e.\ any source file having the name of it's class
in lower-case and extension, holds the definition of non-trivial constructors and
destructors. 
% 
Other member functions, which are too long to be defined in the header file, are
placed in separate files, whose name is formed by the class name, followed by 
the function name. For this example, the only such function is 
{\tt Template::sum\_squares() const;} stored in the file 
{\tt template\_sum\_squares.cpp} outlined bellow:
%
{\small \begin{verbatim}
      1 #include "template.h" /* class header must be included */
      2
      3 /***************************************************************************//**
      4 *  This function mimics a definition of a non-trivial member function.
      5 *******************************************************************************/
      6 real Template :: sum_squares() const {
      7
      8   /*-----------------------------------+
      9   |  compute the result and return it  |
     10   +-----------------------------------*/
     11   return at*at + bt*bt;
     12 }
     13
     14 /*-----------------------------------------------------------------------------+
     15  '$Id: 05-templates.tex,v 1.1 2009/11/12 15:29:43 niceno Exp $'/
     16 +-----------------------------------------------------------------------------*/
\end{verbatim}}
% 

Two important needs to be pointed out:
\begin{itemize} 
  \item No line in any of the source codes should be wider than 80 characters. 
        If it did, it would create broken lines in source files hard-copies,
        and render them difficult to read.
  \item Each files ends with the so called {\em footer}. Footer is sought as a
        three-line comment for {\tt C++}, but contains a quoted line beginning 
        with {\tt \$Id:}. This line is under full control of {\tt CVS}, which
        changes it to insert the revision number, date and time of that 
        revision. This information is very useful when debugging the code.
\end{itemize} 

%---------------------------------------------------------------------nutshell-%
\vspace*{5mm} \fbox{ \begin{minipage}[c] {0.97\textwidth} %-----------nutshell-%
    {\sf Section \ref{sec:templates} in a nutshell} \\ %--------------nutshell-%

      - Class names in {\psiboil}:
      \begin{itemize}
        \item Each class is declared in a separate header file.         
        \item Header file can also contain definition of trivial constructors
              and any other member functions short enough to fit in one line.
        \item Setter and getter functions should be placed in the header.
        \item Constructors longer than one line of code, must be placed in
              a separate file.
        \item No line in any of the sources should be wider than 80 characters.
        \item Each file in the package must end with a footer.          
        \item Template files provided in this section are a good starting
              point when defining new classes.
      \end{itemize}
  \end{minipage} } %--------------------------------------------------nutshell-%
%---------------------------------------------------------------------nutshell-%
