This section explains the implementation of time-stepping 
(time-integration) algorithms in {\psiboil}. Since {\psiboil} supports variety
of time-stepping schemes, this is far from an easy task. To make matters 
even more complex, the implementation spans over several objects and member 
functions. 

Therefore, it is strongly recommended that the reader prints the following 
functions:
%
\begin{itemize}
  \item {\tt Src/Ravioli/timescheme.h}
  \item {\tt Src/Equation/equation.h}
  \item {\tt Src/Equation/Centered/centered.h}
  \item {\tt Src/Equation/Centered/centered\_new\_time\_step.cpp}
  \item {\tt Src/Equation/Centered/centered\_update\_rhs.cpp}
  \item {\tt Src/Equation/Centered/centered\_solve.cpp}
\end{itemize}
%
and keeps the printouts at hand while reading this chapter.

This chapter begins with a recollection of the conservation equation for 
general variable. This is done in order to identify different terms in the
governing equations, their possible time time-discretization schemes and 
relation to the final form of the discretized system of equations.

The section which follows focuses on implementation of these concepts into
{\psiboil}. In order to do that, two objects had to be explained in more
details {\tt TimeScheme} which is a ravioli object holding basic information
on time-stepping schemes and {\tt Centered} which embodies the implementation
of time-stepping schemes. Everything what is explained about the {\tt Centered}
class is also valid for its children, as well as for its sister 
class~{\tt Staggered}. 
