\section{Recollection of some basic concepts}
\label{sec:recollection}

\subsection{General conservation equation}

\noindent
Integral equation for conservation of a general variable reads:
%
\be
  \frac{\p}{\p t} \int_V ( \rho B_\phi \phi ) dV  
  + 
  \oint_S ( \rho B_\phi \uvw \phi ) dS
  = 
  \oint_S  (\Gamma_\phi \nabla \phi ) dS
  +
  \int_V f_\phi dV
\ee
%
If, for the sake of shortness, we introduce the following definitions:
%
\bea
  \mathbb{Q}_\phi & = & \int_V ( \rho B_\phi \phi ) dV         \\ 
  \mathbb{H}_\phi & = & \oint_S ( \rho B_\phi \uvw \phi ) dS   \\ 
  \mathbb{D}_\phi & = & \oint_S  (\Gamma_\phi \nabla \phi ) dS \\
  \mathbb{F}_\phi & = & \int_V f_\phi dV
\eea
%
where $\mathbb{Q}_\phi$ represents the {\em innertial} term, $\mathbb{H}_\phi$ 
the {\em advection}, $\mathbb{D}_\phi$ the {\em diffusion} and 
$\mathbb{F}_\phi$ the {\em forcing} (source) term, the conservation equation 
for a general variable assumes the form:
%
\be
  \frac{\p}{\p t} \mathbb{Q}_\phi + \mathbb{H}_\phi 
  = 
  \mathbb{D}_\phi + \mathbb{F}_\phi
\ee

\subsection{Integration in time}

As the next step, this equation is discretized in time. In order to do that,
it is worthwhile introducing convention on time steps:
\begin{itemize}
  \item {\em New time step}~($N$) is the one which is currently being resolved, 
        i.e.\ the one whose value is unknown. 
  \item {\em Old time step}~($N-1$) represents the time step before new, hence
        it is the last known quantity, and
  \item {\em Very old time step}~($N-2$) is the time step before old.
\end{itemize}
%
In {\psiboil}, a linear variation between the new~($N$) and the old~($N-1$) time 
step is assumed for the innertial term, expressed as:
%
\be
  \frac{\p}{\p t} \mathbb{Q}_\phi 
  = 
  \frac{\mathbb{Q}^N_\phi - \mathbb{Q}^{N-1}_\phi}{\Delta t}
\ee

For advection, diffusion and forcing term different options exist. First order
{\em backward Euler} (fully implicit) discretization for all terms is expressed 
as:
%
\be
  \frac{\mathbb{Q}^N_\phi - \mathbb{Q}^{N-1}_\phi}{\Delta t}
  +
  \mathbb{H}^N_\phi
  =
  \mathbb{D}^N_\phi
  +
  \mathbb{F}^N_\phi.
\ee
%
This scheme is famous for its stability. 
First order {\em forward Euler} (fully explicit) scheme for all terms reads:
%
\be
  \frac{\mathbb{Q}^N_\phi - \mathbb{Q}^{N-1}_\phi}{\Delta t}
  +
  \mathbb{H}^{N-1}_\phi
  =
  \mathbb{D}^{N-1}_\phi
  +
  \mathbb{F}^{N-1}_\phi.
\ee
%
and its most advantages feature is simplicity. Second order, semi-implicit
{\em Crank-Nicolson} for all terms reads:
%
\be
  \frac{\mathbb{Q}^N_\phi - \mathbb{Q}^{N-1}_\phi}{\Delta t}
  +
  \frac{1}{2}\mathbb{H}^{N}_\phi
  +
  \frac{1}{2}\mathbb{H}^{N-1}_\phi
  =
  \frac{1}{2}\mathbb{D}^{N}_\phi
  +
  \frac{1}{2}\mathbb{D}^{N-1}_\phi
  +
  \frac{1}{2}\mathbb{F}^{N}_\phi
  +
  \frac{1}{2}\mathbb{F}^{N-1}_\phi
\ee
%
Very often, different schemes are used for different terms. If advection
terms are treated with {\em Adams-Bashforth}, diffusive with Crank-Nicolson and
forcing terms with forward Euler, the equations would read:
%
\be
  \frac{\mathbb{Q}^N_\phi - \mathbb{Q}^{N-1}_\phi}{\Delta t}
  +
  \frac{3}{2}\mathbb{H}^{N-1}_\phi
  -
  \frac{1}{2}\mathbb{H}^{N-2}_\phi
  =
  \frac{1}{2}\mathbb{D}^{N}_\phi
  +
  \frac{1}{2}\mathbb{D}^{N-1}_\phi
  +
  \mathbb{F}^{N-1}_\phi
\ee
%
This is a very often choice in {\psiboil} simulations, as Adams-Bashforth offers
simplicity for non-linear advection terms, while Crank-Nicolson gives 
stability to linear diffusive terms. 

\subsection{Discretized system of equations}

\noindent
The final result of discretization procedure is a linear system of equations:
%
\be
  [A] \cdot \{\phi^N\} = \{f\}
\ee
%
where $[A]$ represents the system matrix, $\{\phi^N\}$
is the unknown vector at new time step $N$ and $\{f\}$ the right-hand side. 
System matrix contains contribution from inertial and implicit part of the 
diffusion terms, while the right hand side contains everything else: advection 
terms, source terms and explicitly treated diffusive terms. This is illustrated 
in Table~\ref{tbl:system}.
%
\begin{table}[h!]
  \begin{center}
    \begin{tabular}{cc|c|c|}
    \cline{3-3}
    & & \multicolumn{1}{ c|}{Contains contribution from:} \\ 
    \cline{1-3}
    \multicolumn{1}{|c|}{\multirow{3}{*}{Term:}} &
    \multicolumn{1}{ c|}{$[A]$}        & $\mathbb{Q}^N/\Delta t$, 
                                         $\mathbb{D}^N$   \\ 
    \cline{2-3}
    \multicolumn{1}{|c|}{}                        &
    \multicolumn{1}{ c|}{$\{\phi^N\}$} & -         \\ 
    \cline{2-3}
    \multicolumn{1}{|c|}{}                        &
    \multicolumn{1}{ c|}{$\{f\}$}      & $\mathbb{Q}^{N-1}/\Delta t$,
                                         $\mathbb{D}^{N-1}$,    
                                         $\mathbb{H}^{N}$,    
                                         $\mathbb{H}^{N-1}$,    
                                         $\mathbb{H}^{N-2}$,    
                                         $\mathbb{F}$   \\ 
    \cline{1-3}
    \end{tabular}
    \caption{Dependence of linear system of equation on various terms
             in the governing equation.}
    \label{tbl:system}
  \end{center}
\end{table}
%

In Table~\ref{tbl:system} it is asssumed that system matrix does not change 
in time and no assumption on time discretization of the external source 
term~($\mathbb{F}$) is done.
