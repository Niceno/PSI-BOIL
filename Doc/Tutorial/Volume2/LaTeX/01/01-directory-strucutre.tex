\section{Directory structure}
\label{sec:directory_structure}

{\psiboil} has an established directory structure which should be obeyed when
new files are added to the package. 
Once retrieved from the {\tt CVS} database, the {\psiboil}'s directory 
structure (not fully expanded for the sake of clarity), looks like:
%
{\small \begin{verbatim}
PSI-Boil
|-- Benchmarks
|-- Doc
|   |-- Dox
|   |-- Tutorial
|   `-- Various
|-- Jobs
|-- Run
|-- Src
`-- Tests
\end{verbatim}}
%
The content of each sub-directory is as follows:
%
\begin{itemize}
  \item {\tt Benchmarks} -- contains sub-directories with cases to benchmark the 
        code. 
  \item {\tt Doc} -- contains documentation for the code and has tree sub-directories:
  \begin{itemize}
    \item {\tt Dox} -- directory for creating Doxygen\footnote{{\tt www.doxygen.org}} 
          documentation.
    \item {\tt Tutorial} -- \LaTeX sources for this document.
    \item {\tt Various} -- various figures and \LaTeX sources to elucidate certain
          aspects of numerical algorithms used in {\psiboil}. These files are 
          destined to become part of this tutorial or Doxygen documentation. 
  \end{itemize}
  \item {\tt Jobs} -- holds scripts to launch {\psiboil} simulations on various
        computational platforms. 
  \item {\tt Run} -- directory where simulations should be launched.
  \item {\tt Src} -- contains source files. It is elaborated in more detail below.
  \item {\tt Tests} -- similar to {\tt Benchmarks}, but contains cases to check the
        most essential functionality of the code. 
\end{itemize}

The most interesting of these sub-directories, for the present document at least,
is the {\tt Src} which contains the source code for {\psiboil}. When expanded, 
it looks like:
%
{\small \begin{verbatim}
Src
|-- Board
|-- Body
|   `-- Empty
|-- Boundary
|-- Connect
|-- Domain
|-- Equation
|   |-- Centered
|   |   |-- ColorCIP
|   |   |-- Concentration
|   |   |-- Distance
|   |   |-- Enthalpy
|   |   |-- LevelSet
|   |   `-- Pressure
|   `-- Staggered
|       `-- Momentum
|-- Field
|   |-- Scalar
|   `-- Vector
|-- Formula
|-- Global
|-- Grid
|-- Include
|-- Matrix
|-- Matter
|-- Model
|-- Monitor
|   |-- Location
|   `-- Rack
|-- Parallel
|   `-- Out
|-- Plot
|   |-- GMV
|   |-- TEC
|   `-- VTK
|-- Profile
|-- RandomFlow
|-- Ravioli
|-- Solver
|   |-- Additive
|   |-- Gauss
|   |-- Krylov
|   `-- Preconditioner
`-- Timer
\end{verbatim}}
%
Each of these directories contains header (extension {\tt .h}) and source files 
(extension {\tt .cpp}) which define a class with the same name as the directory 
itself, except:
%
\begin{itemize}
  \item {\tt Connect} -- contains a separate program which connects result files 
        after a parallel run.
  \item {\tt Global} and {\tt Ravioli} -- contain definitions of classes which 
        are to small to have a directory of their own. 
\end{itemize}
%
In addition to directories having the same names as the classes they define,
the directory hierarchy itself follows the class hierarchy. For example, class
{\tt Field} has two {\em children}, called {\tt Scalar} and {\tt Vector}. 
Class {\tt Equation} features even more complex structure, having {\tt Centered}
and {\tt Staggered} as it's first ancestors and classes such as {\tt Distance},
{\tt LevelSet}, {\tt Pressure} and {\tt Momentum} as second. 

%---------------------------------------------------------------------nutshell-%
\vspace*{5mm} \fbox{ \begin{minipage}[c] {0.97\textwidth} %-----------nutshell-%
    {\sf Section \ref{sec:directory_structure} in a nutshell} \\ %----nutshell-%

      - {\psiboil} has a clearly defined directory structure:
      \begin{itemize}
        \item There are separate directories for documentation, benchmarks,
              test, runs, batch scripts to run the code and, most important
              of all, with sources to compile the package. 
        \item Directory with sources is divided in separate sub-directories,
              each holding a separate class.
        \item Directory names correspond to class names they hold, with the
              exception of {\tt Connect} (which is a separate program) and
              {\tt Global} and {\tt Ravioli} which contain many small classes.
        \item Hierarchical directory structure closely follows the hierarchy
              of the class inheritance. 
      \end{itemize}
  \end{minipage} } %--------------------------------------------------nutshell-%
%---------------------------------------------------------------------nutshell-%
