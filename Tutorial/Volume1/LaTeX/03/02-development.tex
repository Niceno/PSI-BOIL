\section{Development terminal output}
\label{sec_development}

The objects for terminal output introduced in~\ref{sec_standard} are 
most useful when the program is already well developed, and we want to
enrich its execution with meaningful messages. 

During the program development and debugging, however, we may wish to 
have more detailed messages. Not only the processor i.d.\ would be 
useful, but also the source file from which the message is written,
as well as the line from which it was invoked. {\psiboil} has 
four macros which do just that:
%
\begin{itemize}
  \item {\tt AMS(x)} - {\tt A}ll processors print the {\tt M}e{\tt S}sage {\tt x},
  \item {\tt OMS(x)} - {\tt O}ne processor prints the {\tt M}e{\tt S}sage {\tt x},
  \item {\tt APR(x)} - {\tt A}ll processors {\tt PR}int the value of variable {\tt x},
  \item {\tt OPR(x)} - {\tt O}ne processor {\tt PR}ints the value of variable {\tt x},
\end{itemize}
%
Let's imagine we wanted to obtain the same message as in the last example
of section~\ref{sec_standard} ({\tt Inside the PSI-Boil!}) but in the 
{\em development}-like fashion. We should use the following program: 
%
{\small \begin{verbatim}
      1 #include "Include/psi-boil.h"
      2
      3 /****************************************************************************/
      4 main(int argc, char * argv[]) {
      5
      6   boil::timer.start();
      7
      8   OMS("Inside the PSI-Boil!");
      9
     10   boil::timer.stop();
     11 }
\end{verbatim}}
%
(This program resides in {\tt ../Doc/Tutorial/Volume1/Src/03-05-main.cpp}). 
Following procedures described in previous sections, link this program to
{\tt main.cpp} in the source directory and re-compile it. Do not forget
to remove {\tt main.o} before compilation. 

Once you have the executable (always called {\tt Boil}, residing in source
directory), run it on one processor, and you will get the output: 
%
{\small \begin{verbatim}
FILE: main.cpp, LINE: 8, "Inside the PSI-Boil!"
\end{verbatim}}
%
This is quite useful. It prints the message you want, but it also includes
the file ({\tt main.cpp} in this case) and line number (8 here) from which 
the message is printed.

While developing parallel programs, it is very useful to know which processor
prints the message as well. This is achieved with macro {\tt AMS}. To try it,
change the line~8 to:
%
{\small \begin{verbatim}
      8   AMS("Inside the PSI-Boil!");
\end{verbatim}}
%
(This program is in {\tt ../Doc/Tutorial/Volume1/Src/03-06-main.cpp}). If you
re-compile the program and run it on one processor, you will get:
%
{\small \begin{verbatim}
PROC: 0, FILE: main.cpp, LINE: 8, "Inside the PSI-Boil!"
\end{verbatim}}
%
The same as above, but including the processor number which printed the 
message. As you may expect, of you run the program on eight processors,
you will get:
%
{\small \begin{verbatim}
PROC: 0, FILE: main.cpp, LINE: 8, "Inside the PSI-Boil!"
PROC: 1, FILE: main.cpp, LINE: 8, "Inside the PSI-Boil!"
PROC: 2, FILE: main.cpp, LINE: 8, "Inside the PSI-Boil!"
PROC: 3, FILE: main.cpp, LINE: 8, "Inside the PSI-Boil!"
PROC: 4, FILE: main.cpp, LINE: 8, "Inside the PSI-Boil!"
PROC: 5, FILE: main.cpp, LINE: 8, "Inside the PSI-Boil!"
PROC: 6, FILE: main.cpp, LINE: 8, "Inside the PSI-Boil!"
PROC: 7, FILE: main.cpp, LINE: 8, "Inside the PSI-Boil!"
\end{verbatim}}
%
Needless to say, these kind of messages are important when checking
if all processors reached certain portion of the code. 

While macro's {\tt AMS} and {\tt OMS} are useful for printing only
messages, the remaining two ({\tt APR} and {\tt OPR}) print values of
variables passed to them. Take the following program for example
({\tt ../Doc/Tutorial/Volume1/Src/03-07-main.cpp}):
%
{\small \begin{verbatim}
      1 #include "Include/psi-boil.h"
      2
      3 /****************************************************************************/
      4 main(int argc, char * argv[]) {
      5
      6   boil::timer.start();
      7
      8   int a;
      9
     10   APR(a);
     11
     12   boil::timer.stop();
     13 }
\end{verbatim}}
%
In line~8, new variable {\tt a} is introduced and {\em no} value is assigned 
to it. It's value is printed, in development-like manner, from line~10.
Since the variable is not initialized, the output is in effect system,
compiler, vendor (you name it) dependent, but on my system it looks like:
%
{\small \begin{verbatim}
PROC: 0, FILE: main.cpp, LINE: 10, a = 2326516
PROC: 1, FILE: main.cpp, LINE: 10, a = 10477556
PROC: 2, FILE: main.cpp, LINE: 10, a = 10477556
PROC: 3, FILE: main.cpp, LINE: 10, a = 10477556
PROC: 4, FILE: main.cpp, LINE: 10, a = 10477556
PROC: 5, FILE: main.cpp, LINE: 10, a = 10477556
PROC: 6, FILE: main.cpp, LINE: 10, a = 10477556
PROC: 7, FILE: main.cpp, LINE: 10, a = 10477556
\end{verbatim}}
%
Note that variable {\tt a} has different values on different processors. This
example illustrates two important points:
%
\begin{itemize}
  \item it is very dangerous to use uninitialized variables,
  \item for parallel programs it is particularly dangerous, since uninitialized
        variable might have different values on different processors\footnote{
An object-oriented (OO) enthusiast might argue at this point that one should not
use simple types such as integers, but everything should be defined as an object
since then the object's constructor would be responsible for variable 
initialization. Although it is true, I felt that defining everything as object,
even integers and floating point numbers, could lead to excessive invocation of
their constructors (and destructors) inside various {\psiboil}'s loops
and therefore render an inefficient program. Furthermore, these objects would
presumably hinder compiler optimizations in numerically intensive parts of the 
code (vector-dot and matrix-vector products). Therefore, such an OO-purism has 
no place in numerical simulations.}.
\end{itemize}
%

The usage of the remaining macro ({\tt OPR}) is straightforward and is not covered
in this tutorial.

%---------------------------------------------------------------------nutshell-%
\vspace*{5mm} \fbox{ \begin{minipage}[c] {0.97\textwidth} %-----------nutshell-%
    {\sf Section \ref{sec_development} in a nutshell} \\ %------------nutshell-%

    - {\psiboil} macro's for terminal output useful during the program development 
    are:
    \begin{itemize}
      \item {\tt AMS(x)} - all processors print the message {\tt x},
      \item {\tt OMS(x)} - one processor prints the message {\tt x},
      \item {\tt APR(x)} - all processors print the value of variable {\tt x},
      \item {\tt OPR(x)} - one processor prints the value of variable {\tt x},
    \end{itemize}
    
    - Furthermore, it is worth stressing that:
    \begin{itemize}
      \item it is very dangerous to use uninitialized variables,
      \item which is particularly pronounced for parallel programs.
    \end{itemize}
  \end{minipage} } %--------------------------------------------------nutshell-%
%---------------------------------------------------------------------nutshell-%
