\section{Ravioli classes}
\label{sec_ravioli}

Small objects, such as that of {\tt Range} type, are in {\psiboil} referred to 
as {\em ravioli} objects. They are very small and have limited functionality 
(small number of member functions), but primarily serve for improved code readability, 
safer argument passing and to reduce chances for making coding errors. 

Take class {\tt Periodic}, for example. If it was not defined, one would have
to send some other data types, such as integers, for example. Then, an
integer would be sent to {\tt Grid1D}'s constructor, having value 0 if non-periodic,
and 1 if periodic. Or, maybe, -1 if periodic and 0 if non-periodic. Such
parameters are difficult to understand, and would always have to read 
{\tt Grid1D}'s documentation to recall which number means what (Did we agree
0 was periodic, or non-periodic?). Furthermore, type checking is not ensured
with some simple types, integers and Boolean in particular. If one sends
Boolean variable as a parameter where integer should have been sent, compiler
would not notice it, resulting in hard-to-find bugs. With ravioli parameter,
compiler recognizes, during compilation time, any inconsistencies in parameter
types.  

Most frequently used ravioli classes are all declared in 
directory:~{\tt PSI-Boil/Src/Ravioli}. Those used only inside one other 
class\footnote{For example: type of preconditioner is defined as a ravioli class and is
used only inside the preconditioner class. It would not make sense to place a ravioli
class with such a limited application into~{\tt PSI-Boil/Src/Ravioli} directory.}.

More details about each of the ravioli classes can be obtained either from
it's Doxygen documentation (check the directory:~{\tt PSI-Boil/Doc/Dox}), or
from it's header file. If you find the additional information insufficient,
or even missing, feel free to contact the author. The same goes for all other
classes, of course.

%---------------------------------------------------------------------nutshell-%
\vspace*{5mm} \fbox{ \begin{minipage}[c] {0.97\textwidth} %-----------nutshell-%
    {\sf Sections \ref{sec_one-dimensional} and \ref{sec_ravioli} 
         in a nutshell} \\  %-----------------------------------------nutshell-%
    
      - {\psiboil} supports three-dimensional Cartesian grids, represent with 
      class {\tt Domain}. \\

      - {\tt Domain} is built from three one-dimensional grids 
      ({\tt Grid1D}), defining resolution in $x$, $y$ and $z$ directions. \\

      - To create a 1D grid, use the constructor: 
      \begin{itemize}
        \item {\tt Grid1D(Range<real>(xi\_s, xi\_e), N, Periodic);} 
      \end{itemize}
      for uniform grids, or: 
      \begin{itemize}
        \item {\tt Grid1D(Range<real>(xi\_s, xi\_e), Range<real>(D\_s, D\_e), N, Periodic);} 
      \end{itemize}
      for non-uniform grids. \\

      - {\psiboil} uses quite a lot of small objects, called ravioli, for
      improved code readability, safer argument passing and reduced 
      chance of coding errors. Most of them are declared in sub-directory
      {\tt Ravioli}.
   
  \end{minipage} } %--------------------------------------------------nutshell-%
%---------------------------------------------------------------------nutshell-%
