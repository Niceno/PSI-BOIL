\section{About {\psiboil}}
\label{sec_about}

{\psiboil} ({\sf P}arallel {\sf SI}mulator of {\sf Boil}ing phenomena) is a
three-dimensional, numerical solver for single- and two-phase flows 
with or without heat transfer. It has the capability to simulate conjugate 
heat transfer problems as well. The discretization of governing equations is
based on the~$2^{nd}$ order accurate Finite Volume~(FV) method,
on staggered orthogonal grids. % REF
%
Two-phase flows are simulated with surface tracking algorithm,
based on conservative Level Set~(LS) approach. % REF
%
Linear solvers are based on the Krylov's sub-space family of algorithms,
which can be accelerated with an algebraic multigrid method. % REF

{\psiboil} is written in {\tt C++} programming language and uses Message
Passing Interface ({\tt MPI}) for parallelization. It compiles on most
Linux-based computers\footnote{\tt www.linux.org}. 
The compilation procedure relies on {\tt autotools},
as well as on the {\tt make} utility. On most computational platforms, 
compilation consists of running {\tt configure}, followed by {\tt make}. 
%
{\psiboil} has been compiled on Linux-based PCs, clusters, as
well as main-frame computers such as Cray-XT3\footnote{Cray is
a registered trademark of Cray Inc.\ ({\tt www.cray.com}).}.
It is also possible to compile it on Windows~XP\footnote{Windows~XP
is a registered trademark of Microsoft Inc.\ ({\tt www.microsoft.com}).}.
Cygwin system\footnote{\tt www.cygwin.com} and without~{\tt MPI}
support.

The development of {\psiboil} has started in the summer of~2006,
at Paul Scherrer Institute~(PSI), as an integral part of the project
on Multi-Scale Modeling Analysis~(MSMA), which focuses on mechanistic
modeling of boiling phenomena at multiple scales. Initially, it is
available only to PSI personnel working on the MSMA project. But, should
it prove useful, it might spread further among the academic
community, presumably under a {\tt GNU} license\footnote{www.gnu.org}.
%
The main purpose of {\psiboil} is not to compete with commercial
Computational Fluid Dynamics~(CFD) packages, but rather to serve
as a tool for academic community.

Since it's aim is not to compete with commercial CFD codes, {\psiboil} 
is not an integral program capable to solve a general fluid
flow problem. Rather, it is a suite of objects an algorithms 
which are used as blocks for building programs for particular flow problems. 
As you will see in this tutorial, for every problem being solved, 
a new program\footnote{A new {\tt main()} function, embodied into a
separate~{\tt main.cpp} file.} is written. These programs are supposed
to be short and highly specialized for a particular task. 

Following this rationale, {\psiboil} does not have any input files. 
It's  main program is re-written for each problem being solved and 
actually serves as an input file itself.  
