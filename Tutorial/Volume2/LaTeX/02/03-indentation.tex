\section{Indentation}
\label{sec:indentation}

To be able to keep as much code as possible in the visible part of the 
editor of to avoid over abundant line wrapping, in {\psiboil} an indentation 
of {\bf 2 spaces} is used. Furthermore, {\bf no tabs} are allowed since they 
give the code always a different look depending on the tab settings of the 
original editor. If everything looks nicely lined up with a tab setting of 4 
spaces, it does not look so nicely anymore when the tab setting is changed 
to 3, 5, etc. spaces.

The recommended indentation of two spaces applies to all elements of the code:
from class declaration, function definitions, loops and conditional statements.

\subsection{Class declarations}

For example, declaration of class {\tt CG} begins with:
%
{\small \begin{verbatim}
    14 //////////
    15 //      //
    16 //  CG  //
    17 //      //
    18 //////////
    19 class CG : public Krylov {
    20   public:
    21     CG(const Domain & s, const Prec & pc) : Krylov(s, pc) {allocate(s);}
    22     CG(const Domain & s)                  : Krylov(s)     {allocate(s);}
    23 
\end{verbatim}}
% 
here, lines~14--18 begin at column~0, as well as line~19. Keyword {\tt public},
in line~20, begins at column 2, while the constructor definitions in lines~21
and~22 begin at column~4.

\subsection{Function definitions}

Function definitions are also indented by 2 spaces. The excerpt from the 
constructor for {\tt Scalar} illustrates it:
%
{\small \begin{verbatim}
     3 /******************************************************************************/
     4 Scalar::Scalar(const Domain & d) : alias(false) {
     5 
     6   dom = & d;
     7   allocate(d.ni(), d.nj(), d.nk());
     8   bndcnd = new BndCnd( *dom );
     9 }
\end{verbatim}}
% 
The function definition in line~4 begins at column~0, while the first executable
statement begins at column~2. 

\subsection{{\tt for} loops and {\tt if} conditions}

The indentation of two spaces also applies to loops and if conditions. 
Definition of {\tt CG::solve} function shows it for a {\tt for} loop:
%
{\small \begin{verbatim}
    46   for(int i=0; i<iter; i++) {
    53     prec->solve(M, A, z, r);
    58     rho = r.dot(z);
   108     rho_old = rho;
   109   }
\end{verbatim}}
% 
and {\tt if} condition:
%
{\small \begin{verbatim}
    60     if(i == 0) {
    64       p = z;
    66     } else {
    72       p = z + beta * p;
    73     }
\end{verbatim}}

%---------------------------------------------------------------------nutshell-%
\vspace*{5mm} \fbox{ \begin{minipage}[c] {0.97\textwidth} %-----------nutshell-%
    {\sf Section \ref{sec:indentation} in a nutshell} \\ %-------------nutshell-%

      - Indentation in {\psiboil} boils down to this two rules:
      \begin{itemize}
        \item Two spaces are used for indentation of all programming constructs.
        \item No tabs are allowed.                 
      \end{itemize}
  \end{minipage} } %--------------------------------------------------nutshell-%
%---------------------------------------------------------------------nutshell-%

