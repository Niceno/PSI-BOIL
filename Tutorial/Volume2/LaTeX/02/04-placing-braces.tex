\section{Placing braces}
\label{sec:placing-braces}

As examples in Sec.~\ref{sec:indentation} show, the opening brace is placed 
as the last in the first line of the construct, and closing brace first in
the last line. This convention is followed for all programming constructs.

Some exceptions are allowed. For example, very short function definitions
in header files might have opening and closing brace in the same line.
An example from {\tt domain.h} shows it:
%
{\small \begin{verbatim}
    44     real dxc(const int i) const {return grid_x_local->dxc(i);}
    45     real dxw(const int i) const {return grid_x_local->dxn(i);}
\end{verbatim}}
% 
If such a short function does not fit into one line (it becomes wider 
than 80 characters), the definition is allowed to continue in the second
line, which contains the opening and closing brace:
%
{\small \begin{verbatim}
    97     real dV(const int i, const int j, const int k) const
    98      {return dom->dV(i,j,k);}
\end{verbatim}}
% 
This is allowed to prevent such a short functions to occupy three lines.
The last example is taken from {\tt scalar.h}.

%---------------------------------------------------------------------nutshell-%
\vspace*{5mm} \fbox{ \begin{minipage}[c] {0.97\textwidth} %-----------nutshell-%
    {\sf Section \ref{sec:placing-braces} in a nutshell} \\ %---------nutshell-%

      - For placing of braces in {\psiboil} following two rules must be obeyed:
      \begin{itemize}
        \item Opening brace is last in the first line, and closing brace is     
              first in the last line.
        \item For very short function definitions in the class header, both 
              opening and closing brace can be in the same line.
      \end{itemize}
  \end{minipage} } %--------------------------------------------------nutshell-%
%---------------------------------------------------------------------nutshell-%

