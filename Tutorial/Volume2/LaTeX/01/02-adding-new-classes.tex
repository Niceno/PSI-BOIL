\section{Adding new classes}
\label{sec:adding_new_classes}

Each sub-directory outlined in Section~\ref{sec:directory_structure} contains 
a file called {\tt Makefile.am}, which holds instructions for {\tt autotools} 
how to create {\tt Makefiles}\footnote{A procedure to run the {\tt autotools} 
is outlined in Volume~1 of this tutorial.}.

In addition, a {\tt Makefile.am} file is present in the {\em root} directory
of {\psiboil} (one level above {\tt Src}), but even more important the 
{\tt configure.ac} file, with additional instructions for {\tt autotools}.

Let's imagine, for the sake of the argument, that {\psiboil} does not have
the capability to create results in {\tt VTK} file format, needed for
{\tt Paraview} package~\footnote{{\tt www.paraview.org}}. We need this
ability, so we have to add additional object which has this functionality.

In order to add this object, one must perform the following steps:

\begin{enumerate}
  \item Create a directory where new class will be defined. In case considered
        here, the new directory is: {\tt PSI-Boil/Plot/VTK}\footnote{It is clear 
        that we intend the new class to be inherited from the {\tt Plot}.}. 
  \item Edit the file {\tt PSI-Boil/configure.ac} to add the path to new directory.
        In present case, it is in line~42:
%
{\small \begin{verbatim}
      9 AC_OUTPUT(Makefile                                     \
     10           Src/Domain/Makefile                          \
                  ... 
     38           Src/Model/Makefile                           \
     39           Src/Plot/Makefile                            \
     40           Src/Plot/GMV/Makefile                        \
     41           Src/Plot/TEC/Makefile                        \
     42           Src/Plot/VTK/Makefile                        \
                  ...
\end{verbatim}}
%
  \item Edit the file {\tt PSI-Boil/Src/Makefile.am} to add the path to the new
        object, as well as the new library. For this case, lines~23 and~56 are
        added. 
%
{\small \begin{verbatim}
      6 SUBDIRS = Equation/Staggered/Momentum     \
      7           Equation/Staggered              \
                  ...                               
     19           Domain                          \
     20           Plot                            \
     21           Plot/GMV                        \
     22           Plot/TEC                        \
     23           Plot/VTK                        \
                  ...                               
     39 LDADD = Equation/Staggered/Momentum/libMomentum.a          \
     40         Equation/Staggered/libStaggered.a                  \
                  ...                               
     52         Domain/libDomain.a                                 \
     53         Plot/libPlot.a                                     \
     54         Plot/GMV/libPlotGMV.a                              \
     55         Plot/TEC/libPlotTEC.a                              \
     56         Plot/VTK/libPlotVTK.a                              \
\end{verbatim}}
%
  \item Add a {\tt Makefile.am} into newly created directory {\tt PSI-Boil/Plot/VTK}
        with the following contents:
%
{\small \begin{verbatim}
      1 noinst_LIBRARIES = libPlotVTK.a
      2 libPlotVTK_a_SOURCES = plot_vtk.h \
      3                        plot_vtk.cpp
      4 
      5 INCLUDE = -I@top_srcdir@/
\end{verbatim}}
%
        It is important to take care that the name of the library specified here 
        ({\tt libPlotVTK.a}) is the same as the one specified in the file 
        {\tt PSI-Boil/Src/Makefile.am}, at line~56. In this particular case, we
        assume that the new object will be defined with one header ({\tt plot\_vtk.h}) 
        and source file ({\tt plot\_vtk.cpp}) only.

  \item Add the new files ({\tt plot\_vtk.h} and {\tt plot\_vtk.cpp}) to directory 
        {\tt PSI-Boil/Plot/VTK} and define the new class. 

  \item Finally, the {\tt Makefile}'s for {\psiboil} package have to be re-generated
        using the script {\tt first}, as explained in Volume~1 of this tutorial.

\end{enumerate}

When adding a new class, it is not a bad idea to start from template sources
defined in directory {\tt Src/Template}. These templates are described bellow
in Sec.~\ref{sec:templates}.

%---------------------------------------------------------------------nutshell-%
\vspace*{5mm} \fbox{ \begin{minipage}[c] {0.97\textwidth} %-----------nutshell-%
    {\sf Section \ref{sec:adding_new_classes} in a nutshell} \\ %-----nutshell-%

      - To add a new class to {\psiboil} package, one has to: 
      \begin{itemize}
        \item Create a directory where the new class will reside.
        \item Edit {\tt PSI-Boil/configure.ac} to add a path to the new directory.
        \item Edit {\tt PSI-Boil/Src/Makefile.am} to add a path to the new
              directory and specify new library name.
        \item Create {\tt Makefile.am} in the new class' directory.
        \item Crate sources in the new class' directory.
        \item Re-run {\tt autotools} to create new {\tt Makefile}'s.
      \end{itemize}
  \end{minipage} } %--------------------------------------------------nutshell-%
%---------------------------------------------------------------------nutshell-%
