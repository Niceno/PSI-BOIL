\section{Implementation in {\psiboil}}
\label{sec:implementation}

\subsection{Storing information on time-stepping scheme}

In a computer program one needs a way to impose and control the time stepping 
schemes. In {\psiboil}, it is achieved with the ravioli object {\tt TimeScheme} 
defined in {\tt Src/Ravioli}. From {\em user} point of view, one adjusts the 
desired time stepping scheme in the main function. 
For example, program ({\tt 07-01-main.cpp}) from Volume~1 of the tutorial, 
defines the enthalpy conservation equation and sets the time stepping scheme
for diffusion as:
%
{\small \begin{verbatim}
     29
     30   Enthalpy enth(t, q, uvw, time, solver, &solid);   /* enthalpy conservation
     31                                                        equation              */
     32   enth.diffusion_set(TimeScheme::backward_euler()); /* time stepping scheme */
     33
\end{verbatim}}
%
A {\tt TimeScheme} object has three member functions: {\tt N()}, {\tt Nm1()} 
and~{\tt Nm2()} which hold the coefficients multiplying time levels at $N$, 
$N-1$ and~$N-2$ respectively.
%
Table~\ref{tbl:schemes} illustrates this point by summarizing values member
functions return, depending on the time-stepping scheme.
%
\begin{table}[h!]
  \begin{center}
    \begin{tabular}{rr|c|c|c|c|l}
      \cline{3-5}
        &  & \multicolumn{3}{ c|}{Function:}       \\ 
      \cline{3-5}
        &  & {\tt N()} & {\tt Nm1()} & {\tt Nm2()} \\ 
      \cline{1-5}
      \multicolumn{1}{|c|}{\multirow{4}{*}{\tt TimeScheme::}} &
      \multicolumn{1}{ c|}{\tt backward\_euler()}  &  1  &  0  &  0   \\ \cline{2-5}
      \multicolumn{1}{|c|}{}                   &
      \multicolumn{1}{ c|}{\tt forward\_euler()}   &  0  &  1  &  0   \\ \cline{2-5}
      \multicolumn{1}{|c|}{}                   &
      \multicolumn{1}{ c|}{\tt crank\_nicolson()}  & 1/2 & 1/2 &  0   \\ \cline{2-5}
      \multicolumn{1}{|c|}{}                   &
      \multicolumn{1}{ c|}{\tt adams\_bashforth()} &  0  & 3/2 &-1/2   \\ \cline{1-5}
    \end{tabular}
    \caption{Values of member functions {\tt N()}, {\tt Nm1()} and~{\tt Nm2()} for 
             different time-stepping schemes.}
    \label{tbl:schemes}
  \end{center}
\end{table}

\subsection{Implementation of the time-stepping algorithms}

\noindent
From {\em development} point of view, {\tt TimeScheme}s are contained in object 
{\tt Equation}, as shown with an excerpt from the file 
{\tt equation.h}:
%
{\small \begin{verbatim}
     42
     43   protected:
     44     TimeScheme diff_ts;
     45     TimeScheme conv_ts;
\end{verbatim}}
%
So, there are two objects of type {\tt TimeScheme} defined inside {\tt Equation}
called {\tt conv\_ts} and {\tt diff\_ts} which define time-stepping schemes for
advection and diffusion terms. Their default values are set in the constructor,
as shown in lines 26--27 of the same file:
%
{\small \begin{verbatim}
     18 class Equation {
     19   public:
     20     Equation(const Domain * d,
     21              const Times  * t,
     22              Matter * f,
     23              Matter * s,
     24              Krylov * sm) :
     25       dom(d), time(t), flu(f), sol(s), solver(sm),
     26       conv_ts(TimeScheme::adams_bashforth()),
     27       diff_ts(TimeScheme::crank_nicolson()),
     28       lim(ConvScheme::superbee()) {}
\end{verbatim}}
%
As this constructor shows, the default scheme for advection is Adams-Bashforth
and for diffusion is Crank-Nicolson.  

Objects {\tt Centered} and {\tt Staggered} store parts of the linear system of
equations in: {\tt Matrix A} which holds the system matrix~$[A]$, {\tt phi} 
holding the variable~$\{\phi\}$ and {\tt Scalar fnew} holding the right hand
side~$\{f\}$. In addition, {\tt Scalars} {\tt fold}, {\tt cnew}, {\tt cold}
are used for storing intermediate parts of right-hand side.

\subsubsection{Non-iterative time-advancement}

Let's take the program {\tt 09-02-main.cpp} from the Volume~1 tutorial and
focus only on the solution of enthalpy equation. A fraction of the 
{\tt 09-02-main.cpp} is included below:
%
{\small \begin{verbatim}
     77   Enthalpy enth( t,   g,   uvw, time, solver, &fluid);
     ...
     83   for(time.start(); time.end(); time.increase()) {
     ...
     93     enth.new_time_step();
     94     enth.solve(ResRat(0.001));
     ...
\end{verbatim}}
%
Line 77 defines the equation for enthalpy, line~83 starts the time loop and
lines~93 and~94 are all that is needed to advance enthalpy in time for
non-iterative algorithms. 
Since the time-stepping schemes was not changed for {\tt Enthalpy}, it uses the 
default settings which is Adams-Bashforth for advection and Crank-Nicolson for 
diffusion.
%
When a new time step begins (after line~83), data members from {\tt Centered} 
contain the following:
%
  \begin{center}
    \begin{tabular}{cc|c|cc}
    \cline{3-3}
    & & \multicolumn{1}{ c|}{Holds:} & \\ 
    \cline{1-3}
    \multicolumn{1}{|c|}{\multirow{3}{*}{Member:}} &
    \multicolumn{1}{ l|}{\tt Matrix A}   & $[A]$ &      \\ 
    \cline{2-3}
    \multicolumn{1}{|l|}{} &
    \multicolumn{1}{ l|}{\tt Scalar phi} & $\{\phi^{N-1}\}$ & \\ 
    \cline{2-3}
    \multicolumn{1}{|l|}{} &
    \multicolumn{1}{ l|}{\tt Scalar cold} & $\mathbb{H}^{N-2} $ & \\
    \cline{1-3}
    \end{tabular}
  \end{center}
%
The reason why {\tt cold} contains $\mathbb{H}^{N-2}$ will be clear by the 
end of this section.

With these values we enter {\tt Centered::new\_time\_step()} function, defined 
in {\tt centered\_new\_time\_step.cpp}, outlined below:
%
{\small \begin{verbatim}
     ...
     13 void Centered::new_time_step() {
     ...
     23   /* no transport in solid */
     24   if( !solid() )
     25     for_ijk(i,j,k) {
     26       const real c = fluid()->cp (i,j,k);
     27       const real r = fluid()->rho(i,j,k);
     28
     29       fold[i][j][k] = r * c * dV(i,j,k) * phi[i][j][k] * time->dti()
     30                     + conv_ts.Nm2() * cold[i][j][k];  
     31     }
     ...
     53   /* a condition like: if(conv_ts != backward_euler()) would be good */
     54   convection(&cold);
     55   for_ijk(i,j,k)
     56     fold[i][j][k] += conv_ts.Nm1() * cold[i][j][k]; /* conv_ts.Nm1() = 1.5 */
     ...
     63   /* a condition like: if(diff_ts != backward_euler()) would be good */
     64   diffusion();
     ...
\end{verbatim}}
%
Lines 25--31 update the right-hand side with the part of the unsteady term from
time step~$N-1$ and advection term from time step $N-2$. After line~31, data 
members hold the following:
%
  \begin{center}
    \begin{tabular}{cc|c|cc}
    \cline{3-3}
    & & \multicolumn{1}{ c|}{Holds:} & \\ 
    \cline{1-3}
    \multicolumn{1}{|c|}{\multirow{4}{*}{Member:}} &
    \multicolumn{1}{ l|}{\tt Matrix A}   & $[A]$ &      \\ 
    \cline{2-3}
    \multicolumn{1}{|l|}{} &
    \multicolumn{1}{ l|}{\tt Scalar phi} & $\{\phi^{N-1}\}$ & \\ 
    \cline{2-3}
    \multicolumn{1}{|l|}{} &
    \multicolumn{1}{ l|}{\tt Scalar fold} & $\mathbb{Q}^{N-1}/\Delta t$,
                                            $\mathbb{H}^{N-2} $ & $\gets$ change \\
    \cline{2-3}
    \multicolumn{1}{|l|}{} &
    \multicolumn{1}{ l|}{\tt Scalar cold} & $\mathbb{H}^{N-2} $ & \\
    \cline{1-3}
    \end{tabular}
  \end{center}
%
Line~54 calls member function {\tt convection} which computes the advection
term with the last available velocity and stores it into member {\tt cnew}.
In the present case, last available velocity is the one from the previous
time step ($N-1$), so function {\tt convection} will 
compute~$\mathbb{H}^{N-1}$. Therefore, after the line~54, data members have 
contain the following:
%
  \begin{center}
    \begin{tabular}{cc|c|cc}
    \cline{3-3}
    & & \multicolumn{1}{ c|}{Holds:} & \\ 
    \cline{1-3}
    \multicolumn{1}{|c|}{\multirow{4}{*}{Member:}} &
    \multicolumn{1}{ l|}{\tt Matrix A}   & $[A]$ &      \\ 
    \cline{2-3}
    \multicolumn{1}{|l|}{} &
    \multicolumn{1}{ l|}{\tt Scalar phi} & $\{\phi^{N-1}\}$ & \\ 
    \cline{2-3}
    \multicolumn{1}{|l|}{} &
    \multicolumn{1}{ l|}{\tt Scalar fold} & $\mathbb{Q}^{N-1}/\Delta t$,
                                            $\mathbb{H}^{N-2} $ & \\
    \cline{2-3}
    \multicolumn{1}{|l|}{} &
    \multicolumn{1}{ l|}{\tt Scalar cold} & $\mathbb{H}^{N-1} $ & $\gets$ change \\
    \cline{1-3}
    \end{tabular}
  \end{center}
%
Lines~55 and~56 update the right hand side with advection from previous time
step. Thus, after line 56 data members hold the following:
%
  \begin{center}
    \begin{tabular}{cc|c|cc}
    \cline{3-3}
    & & \multicolumn{1}{ c|}{Holds:} & \\ 
    \cline{1-3}
    \multicolumn{1}{|c|}{\multirow{4}{*}{Member:}} &
    \multicolumn{1}{ l|}{\tt Matrix A}   & $[A]$ &      \\ 
    \cline{2-3}
    \multicolumn{1}{|l|}{} &
    \multicolumn{1}{ l|}{\tt Scalar phi} & $\{\phi^{N-1}\}$ & \\ 
    \cline{2-3}
    \multicolumn{1}{|l|}{} &
    \multicolumn{1}{ l|}{\tt Scalar fold} & $\mathbb{Q}^{N-1}/\Delta t$,
                                            $\mathbb{H}^{N-2}$,
                                            $\mathbb{H}^{N-1}$ & $\gets$ change \\
    \cline{2-3}
    \multicolumn{1}{|l|}{} &
    \multicolumn{1}{ l|}{\tt Scalar cold} & $\mathbb{H}^{N-1} $ & \\
    \cline{1-3}
    \end{tabular}
  \end{center}
%
Line~64 calls member function {\tt diffusion} which updates {\tt fold}
with the explicitly treated part of the diffusion term. After line~64
data members have the following meaning:
%
  \begin{center}
    \begin{tabular}{cc|c|cc}
    \cline{3-3}
    & & \multicolumn{1}{ c|}{Holds:} & \\ 
    \cline{1-3}
    \multicolumn{1}{|c|}{\multirow{4}{*}{Member:}} &
    \multicolumn{1}{ l|}{\tt Matrix A}   & $[A]$ &      \\ 
    \cline{2-3}
    \multicolumn{1}{|l|}{} &
    \multicolumn{1}{ l|}{\tt Scalar phi} & $\{\phi^{N-1}\}$ & \\ 
    \cline{2-3}
    \multicolumn{1}{|l|}{} &
    \multicolumn{1}{ l|}{\tt Scalar fold} & $\mathbb{Q}^{N-1}/\Delta t$,
                                            $\mathbb{H}^{N-2}$,
                                            $\mathbb{H}^{N-1}$,
                                            $\mathbb{D}^{N-1}$ & $\gets$ change \\
    \cline{2-3}
    \multicolumn{1}{|l|}{} &
    \multicolumn{1}{ l|}{\tt Scalar cold} & $\mathbb{H}^{N-1} $ & \\
    \cline{1-3}
    \end{tabular}
  \end{center}
%
Data members hold the same meaning when exiting from the {\tt new\_time\_step}
function. After the call to {\tt new\_time\_step} in the main function, a call
to {\tt solve} is issued. Part of the {\tt Centered::solve()}, defined in
{\tt centered\_solve.cpp}, is given below:
%
{\small \begin{verbatim}
     ...
      9 void Centered::solve(const real & fact, const char * name) {
     ...
     23   /* solve */
     24   update_rhs();
     25   solver->solve(A, phi, fnew, 20, name);
     ...
\end{verbatim}}
%
and shows that a call to {\tt Centered::update\_rhs()} is performed in line~24,
just before staring the linear solver in line~25. As its name implies, this 
function updates the right-hand side, as the portion of the code, located in
{\tt centered\_update\_rhc.cpp} shows:
%
{\small \begin{verbatim}
     12 real Centered::update_rhs() {
     ...
     19   for_ijk(i,j,k) {
     20     fnew[i][j][k] = fold[i][j][k]
     21                   + cnew[i][j][k] * conv_ts.N()
     22                   + fbnd[i][j][k]
     23                   + fext[i][j][k];
     24   }
     ...
\end{verbatim}}
%
As lines~19--24 show, this function fills the right hand side with all the 
terms needed. After the {\tt Centered::update\_rhs()}, which is also just
before calling the liner solver at line~25 in {\tt Centered::solve()}, 
data members hold the following:
%
  \begin{center}
    \begin{tabular}{cc|c|cc}
    \cline{3-3}
    & & \multicolumn{1}{ c|}{Holds:} & \\ 
    \cline{1-3}
    \multicolumn{1}{|c|}{\multirow{4}{*}{Member:}} &
    \multicolumn{1}{ l|}{\tt Matrix A}   & $[A]$ &      \\ 
    \cline{2-3}
    \multicolumn{1}{|l|}{} &
    \multicolumn{1}{ l|}{\tt Scalar phi} & $\{\phi^{N-1}\}$ & \\ 
    \cline{2-3}
    \multicolumn{1}{|l|}{} &
    \multicolumn{1}{ l|}{\tt Scalar fnew} & $\mathbb{Q}^{N-1}/\Delta t$,
                                            $\mathbb{H}^{N-2}$,
                                            $\mathbb{H}^{N-1}$,
                                            $\mathbb{D}^{N-1}$,  
                                            $\mathbb{F}$ & $\gets$ change \\
    \cline{2-3}
    \multicolumn{1}{|l|}{} &
    \multicolumn{1}{ l|}{\tt Scalar fold} & $\mathbb{Q}^{N-1}/\Delta t$,
                                            $\mathbb{H}^{N-2}$,
                                            $\mathbb{H}^{N-1}$,
                                            $\mathbb{D}^{N-1}$ & \\
    \cline{2-3}
    \multicolumn{1}{|l|}{} &
    \multicolumn{1}{ l|}{\tt Scalar cold} & $\mathbb{H}^{N-1} $ & \\
    \cline{1-3}
    \end{tabular}
  \end{center}
%
Note that, although line~21 in {\tt centered\_update\_rhc.cpp}
adds contribution from {\tt cnew}, it is also multiplied by {\tt conv\_ts.N()},
which is equal to zero for non-iterative time-stepping schemes. 

At this stage, {\tt fnew} holds all the contributions it needs and a call to 
linear solver can be performed, which yields new value of dependent variable.
Hence, after the call to solver, data members contain the following:
%
  \begin{center}
    \begin{tabular}{cc|c|cc}
    \cline{3-3}
    & & \multicolumn{1}{ c|}{Holds:} & \\ 
    \cline{1-3}
    \multicolumn{1}{|c|}{\multirow{5}{*}{Member:}} &
    \multicolumn{1}{ l|}{\tt Matrix A}   & $[A]$ &      \\ 
    \cline{2-3}
    \multicolumn{1}{|l|}{} &
    \multicolumn{1}{ l|}{\tt Scalar phi} & $\{\phi^N\}$ & $\gets$ change \\ 
    \cline{2-3}
    \multicolumn{1}{|l|}{} &
    \multicolumn{1}{ l|}{\tt Scalar fnew} & $\mathbb{Q}^{N-1}/\Delta t$,
                                            $\mathbb{H}^{N-2}$,
                                            $\mathbb{H}^{N-1}$,
                                            $\mathbb{D}^{N-1}$,  
                                            $\mathbb{F}$ &  \\
    \cline{2-3}
    \multicolumn{1}{|l|}{} &
    \multicolumn{1}{ l|}{\tt Scalar fold} & $\mathbb{Q}^{N-1}/\Delta t$,
                                            $\mathbb{H}^{N-2}$,
                                            $\mathbb{H}^{N-1}$,
                                            $\mathbb{D}^{N-1}$ & \\
    \cline{2-3}
    \multicolumn{1}{|l|}{} &
    \multicolumn{1}{ l|}{\tt Scalar cold} & $\mathbb{H}^{N-1} $ & \\
    \cline{1-3}
    \end{tabular}
  \end{center}
%
The first three members outlined above ({\tt Matrix A}, {\tt Scalar phi} and
{\tt Scalar fnew}) contain the same entries outlined in Table~\ref{tbl:system},
meaning the linear system for the governing equation is discretized properly. 

As soon as the computation of new time step starts, values in~{\tt phi} and 
{\tt cold} get older by one time level and become~$\{\phi^{N-1}\}$ 
and~$\mathbb{H}^{N-2}$ respectively. That explains the contents of the first
table in this section. 

\subsubsection{Iterative time-advancement}

The algorithm outlined above works for non-iterative time-integration algorithms
such as the projection method. For some simulations an iterative algorithm is 
required for stronger coupling of momentum and pressure, of for implicit 
treatment of non-linear or source terms. The most wide-spread iterative 
time-stepping scheme is the SIMPLE algorithm. An example of a SIMPLE-like
iterative scheme is given in the file {\tt Tests/simple/main-simple.cpp}.
A portion of the file pertinent to time-integration is given below:
%
{\small \begin{verbatim}
     ...
     96   for(time.start(); time.end(); time.increase()) {
     ...
    106     /* momentum */
    107     ns.cfl_max();
    108     ns.new_time_step();
    109
    110     // NEW: SIMPLE ITERATIONS
    111     for(int in=0; in<6; in++) {
     ...                         
    113       ns.convection();
    114       ns.solve(ResRat(0.0001));
    ...
\end{verbatim}}
%
In this case, we are solving momentum equations, rather than enthalpy. But that
doesn't matter, because objects {\tt Enthalpy} and {\tt Momentum} both descend
from {\tt Equation} and concepts for time integration are the same.

Iterative algorithms feature the so-called {\em outer} and {\em inner} loop. For
the present example, outer loop begins at line~96, whereas the inner begins at
line~111. Function {\tt Staggered::new\_time\_step()} is the analog of its 
sister variant {\tt Centered::new\_time\_step()}, and the local data member have 
the following meaning at the end of line~108:
%
  \begin{center}
    \begin{tabular}{cc|c|c|}
    \cline{3-3}
    & & \multicolumn{1}{ c|}{Holds:} \\ 
    \cline{1-3}
    \multicolumn{1}{|c|}{\multirow{5}{*}{Member:}} &
    \multicolumn{1}{ l|}{\tt Matrix A}   & $[A]$       \\ 
    \cline{2-3}
    \multicolumn{1}{|l|}{} &
    \multicolumn{1}{ l|}{\tt Scalar phi} & $\{\phi^{N-1}\}$  \\ 
    \cline{2-3}
    \multicolumn{1}{|l|}{} &
    \multicolumn{1}{ l|}{\tt Scalar fnew} & $\mathbb{Q}^{N-1}/\Delta t$,
                                            $\mathbb{H}^{N-2}$,
                                            $\mathbb{H}^{N-1}$,
                                            $\mathbb{D}^{N-1}$,  
                                            $\mathbb{F}$ \\
    \cline{2-3}
    \multicolumn{1}{|l|}{} &
    \multicolumn{1}{ l|}{\tt Scalar fold} & $\mathbb{Q}^{N-1}/\Delta t$,
                                            $\mathbb{H}^{N-2}$,
                                            $\mathbb{H}^{N-1}$,
                                            $\mathbb{D}^{N-1}$ \\
    \cline{2-3}
    \multicolumn{1}{|l|}{} &
    \multicolumn{1}{ l|}{\tt Scalar cold} & $\mathbb{H}^{N-1} $ \\
    \cline{1-3}
    \end{tabular}
  \end{center}
%
At this point, we enter the {\em inner} loop which contains a call to 
{\tt Momentum::convection()} at line~113. This subroutine updates {\tt cnew} 
with the advection term obtained from new velocities, i.e. velocities currently 
being iterated at time step~$N$. Data members in {\tt Momentum} now contain 
the following:
%
  \begin{center}
    \begin{tabular}{cc|c|cc}
    \cline{3-3}
    & & \multicolumn{1}{ c|}{Holds:} & \\ 
    \cline{1-3}
    \multicolumn{1}{|c|}{\multirow{6}{*}{Member:}} &
    \multicolumn{1}{ l|}{\tt Matrix A}   & $[A]$ &      \\ 
    \cline{2-3}
    \multicolumn{1}{|l|}{} &
    \multicolumn{1}{ l|}{\tt Scalar phi} & $\{\phi\}^{N-1}$  & \\ 
    \cline{2-3}
    \multicolumn{1}{|l|}{} &
    \multicolumn{1}{ l|}{\tt Scalar fnew} & $\mathbb{Q}^{N-1}/\Delta t$,
                                            $\mathbb{H}^{N-2}$,
                                            $\mathbb{H}^{N-1}$,
                                            $\mathbb{D}^{N-1}$,  
                                            $\mathbb{F}$ & \\
    \cline{2-3}
    \multicolumn{1}{|l|}{} &
    \multicolumn{1}{ l|}{\tt Scalar fold} & $\mathbb{Q}^{N-1}/\Delta t$,
                                            $\mathbb{H}^{N-2}$,
                                            $\mathbb{H}^{N-1}$,
                                            $\mathbb{D}^{N-1}$ & \\
    \cline{2-3}
    \multicolumn{1}{|l|}{} &
    \multicolumn{1}{ l|}{\tt Scalar cnew} & $\mathbb{H}^{N}   $ & $\gets$ change \\
    \cline{2-3}
    \multicolumn{1}{|l|}{} &
    \multicolumn{1}{ l|}{\tt Scalar cold} & $\mathbb{H}^{N-1} $ & \\
    \cline{1-3}
    \end{tabular}
  \end{center}
%
In the call to solver at line~114, right hand side {\tt fnew} is updated with
{\tt cnew}, so that the contents of inner variables before the call to linear
solver is:
%
  \begin{center}
    \begin{tabular}{cc|c|cc}
    \cline{3-3}
    & & \multicolumn{1}{ c|}{Stores:} & \\ 
    \cline{1-3}
    \multicolumn{1}{|c|}{\multirow{6}{*}{Member:}} &
    \multicolumn{1}{ l|}{\tt Matrix A}   & $[A]$ &      \\ 
    \cline{2-3}
    \multicolumn{1}{|l|}{} &
    \multicolumn{1}{ l|}{\tt Scalar phi} & $\{\phi^{N-1}\}$ & \\ 
    \cline{2-3}
    \multicolumn{1}{|l|}{} &
    \multicolumn{1}{ l|}{\tt Scalar fnew} & $\mathbb{Q}^{N-1}/\Delta t$,
                                            $\mathbb{H}^{N-2}$,
                                            $\mathbb{H}^{N-1}$,
                                            $\mathbb{H}^{N}$,
                                            $\mathbb{D}^{N-1}$,  
                                            $\mathbb{F}$ & $\gets$ changed \\
    \cline{2-3}
    \multicolumn{1}{|l|}{} &
    \multicolumn{1}{ l|}{\tt Scalar fold} & $\mathbb{Q}^{N-1}/\Delta t$,
                                            $\mathbb{H}^{N-2}$,
                                            $\mathbb{H}^{N-1}$,
                                            $\mathbb{D}^{N-1}$ & \\
    \cline{2-3}
    \multicolumn{1}{|l|}{} &
    \multicolumn{1}{ l|}{\tt Scalar cnew} & $\mathbb{H}^{N}   $ & \\
    \cline{2-3}
    \multicolumn{1}{|l|}{} &
    \multicolumn{1}{ l|}{\tt Scalar cold} & $\mathbb{H}^{N-1} $ & \\
    \cline{1-3}
    \end{tabular}
  \end{center}
%
which is exactly what is needed. After the call to linear solver, {\tt phi}
will hold the new value for velocities $\{\phi^N\}$, i.e.:
%
  \begin{center}
    \begin{tabular}{cc|c|cc}
    \cline{3-3}
    & & \multicolumn{1}{ c|}{Stores:} & \\ 
    \cline{1-3}
    \multicolumn{1}{|c|}{\multirow{6}{*}{Member:}} &
    \multicolumn{1}{ l|}{\tt Matrix A}   & $[A]$ &      \\ 
    \cline{2-3}
    \multicolumn{1}{|l|}{} &
    \multicolumn{1}{ l|}{\tt Scalar phi} & $\{\phi^N\}$ & $\gets$ changed \\ 
    \cline{2-3}
    \multicolumn{1}{|l|}{} &
    \multicolumn{1}{ l|}{\tt Scalar fnew} & $\mathbb{Q}^{N-1}/\Delta t$,
                                            $\mathbb{H}^{N-2}$,
                                            $\mathbb{H}^{N-1}$,
                                            $\mathbb{H}^{N}$,
                                            $\mathbb{D}^{N-1}$,  
                                            $\mathbb{F}$ & \\
    \cline{2-3}
    \multicolumn{1}{|l|}{} &
    \multicolumn{1}{ l|}{\tt Scalar fold} & $\mathbb{Q}^{N-1}/\Delta t$,
                                            $\mathbb{H}^{N-2}$,
                                            $\mathbb{H}^{N-1}$,
                                            $\mathbb{D}^{N-1}$ & \\
    \cline{2-3}
    \multicolumn{1}{|l|}{} &
    \multicolumn{1}{ l|}{\tt Scalar cnew} & $\mathbb{H}^{N}   $ & \\
    \cline{2-3}
    \multicolumn{1}{|l|}{} &
    \multicolumn{1}{ l|}{\tt Scalar cold} & $\mathbb{H}^{N-1} $ & \\
    \cline{1-3}
    \end{tabular}
  \end{center}
%

When the computation of new time step starts, values in~{\tt phi} and 
{\tt cold} get older by one time level and become~$\{\phi^{N-1}\}$ 
and~$\mathbb{H}^{N-2}$ respectively. 

\subsubsection{A note on external source terms}

The way external sources (or forces in case of momentum equation) are 
discretized in time is defined implicitly in the main function. 
If sources are estimated
at the beginning of time step using the know values from old~($N-1$) time
step, their discretization is based on forward Euler scheme. That is the 
most straight-forward approach for non-iterative schemes. 

In case iterative schemes are used, we can compute the value of the source
term inside the inner loop and thus have it based on the new~($N$) value of 
dependent variable making it a backward Euler scheme.

For multi-level schemes (Adams-Bashforth or Crank-Nicolson) one would have to
define two time level for sources in the main function and take care of their
proper computation. At the time of writing this tutorial, there are no examples
to demonstrate such an approach.

%---------------------------------------------------------------------nutshell-%
\vspace*{5mm} \fbox{ \begin{minipage}[c] {0.97\textwidth} %-----------nutshell-%
    {\sf Section \ref{sec:implementation} in a nutshell} \\ %-----nutshell-%

      - Implementation of time-stepping schemes in {\psiboil} is based on the
      following concepts:
      \begin{itemize}
        \item available time-discretization schemes are stored in objects of 
              type {\tt TimeScheme},
        \item {\em inertial} term of the governing equations is assumed to 
              vary linearly between the old ($N-1$) and the new ($N$) time step,
        \item {\em diffusion} term can be discretized in time with the 
              Crank-Nicolson, backward Euler and forward Euler schemes,
        \item {\em advection} term can be discretized in time with the 
              Adams-Bashforth, Crank-Nicolson, backward Euler and forward Euler 
              schemes,
        \item time-stepping scheme for {\em source} (or {\em force} term for 
              momentum) is done in the {\psiboil}'s main function, 
        \item objects of type {\tt Equation} and its descendants, keep various
              parts of discretized governing equations in {\tt Scalar} fields:
              {\tt fold}, {\tt fnew}, {\tt cold}, {\tt cnew},
        \item these {\tt Scalar} fields are updated across several {\tt Equation}
              descendant member functions, most notably {\tt new\_time\_step()},
              {\tt convection}, {\tt diffusion}, {\tt update\_rhs()} and {\tt solve}.
      \end{itemize}
      
  \end{minipage} } %--------------------------------------------------nutshell-%
%---------------------------------------------------------------------nutshell-%
