% This is a sample LaTeX input file.  (Version of 9 April 1986) 
% 
% A '%' character causes TeX to ignore all remaining text on the line, 
% and is used for comments like this one. 

\documentclass[fleqn]{article}    % Specifies the document style. 

\usepackage{graphicx}
\usepackage{psfrag}

%-------------------------
% begin and end equations
%-------------------------
\newcommand{\be} {\begin{equation}}
\newcommand{\ee} {\end{equation}}
\newcommand{\bea}{\begin{eqnarray}}
\newcommand{\eea}{\end{eqnarray}}
\newcommand{\ba} {\begin{array}}
\newcommand{\ea} {\end{array}}

%-----------------
% text formatting 
%-----------------
\newcommand{\noi}{\noindent}

%---------------
% special signs
%---------------
\newcommand{\p}  {\partial}
\newcommand{\s}  {\star}
\newcommand{\Dx} {\Delta}

%-----------------
% velocity vector
%-----------------
\newcommand{\uvw}{{\bf u}}          

%---------------------------
% putting things up or down
%---------------------------
\newcommand{\ol} {\overline}
\newcommand{\ul} {\underline}

%---------------------
% logical coordinates
%---------------------
\newcommand{\ijk}     { {i,j,k} }      % central

\newcommand{\ipjk}    { {i+1,j,k} }    % i+1
\newcommand{\imjk}    { {i-1,j,k} }    % i-1
\newcommand{\ijpk}    { {i,j+1,k} }    % j+1
\newcommand{\ijmk}    { {i,j-1,k} }    % j-1
\newcommand{\ijkp}    { {i,j,k+1} }    % k+1
\newcommand{\ijkm}    { {i,j,k-1} }    % k-1

\newcommand{\ijpkp}   { {i,j+1,k+1} }  % i
\newcommand{\ijpkm}   { {i,j+1,k-1} }
\newcommand{\ijmkp}   { {i,j-1,k+1} }
\newcommand{\ijmkm}   { {i,j-1,k-1} }

\newcommand{\ipjkp}   { {i+1,j,k+1} }  % j
\newcommand{\ipjkm}   { {i+1,j,k-1} }
\newcommand{\imjkp}   { {i-1,j,k+1} }
\newcommand{\imjkm}   { {i-1,j,k-1} }

\newcommand{\ipjpk}   { {i+1,j+1,k} }  % k
\newcommand{\ipjmk}   { {i+1,j-1,k} }
\newcommand{\imjpk}   { {i-1,j+1,k} }
\newcommand{\imjmk}   { {i-1,j-1,k} }

%--------------------------------------
% compute length for framing equations
%--------------------------------------
\newlength{\ii}
\setlength{\ii}{\textwidth}
\addtolength{\ii}{-2\fboxsep}
\addtolength{\ii}{-2\fboxrule}

% To frame an equation, use:
% \begin{center} \fbox{ \begin{minipage}{\ii} \be
% ... equation ... 
% \ee \end{minipage} } \end{center}

%-------------------
% to include figures
%-------------------
\usepackage{graphicx}
\usepackage{psfrag}
% \newcommand{\thickbox}[2]{\thicklines\put(0,0){\framebox(#1,#2){}}}
\newcommand{\thickbox}[2]{}

%----------------------------
% fancy mathematical symbols
%----------------------------
\usepackage{amssymb}

%---------------
% PSI-Boil sign
%---------------
\newcommand{\psiboil}{\sffamily PSI-Boil}


\begin{document}           % End of preamble and beginning of text. 

%============================================%
%                                            %
%  Conservative Form of Governing Equations  %
%                                            %
%============================================%
\section{Conservative Form of Governing Equations}

\noindent
The conservative form of governing equations reads:
%
\be
    \frac{\p (\rho \phi)}{\p t} 
  + \nabla ( \rho {\bf u} \phi ) 
  = D + S.
  \label{eq:conservative}
\ee
%
But this is not the form we solve in {\psiboil}. 
It wouldn't work, it would diverge.
We actually solve the following form:
%
\be 
    \rho \frac{\p \phi}{\p t} 
  + \rho \nabla ( {\bf u} \phi ) 
  = D + S,
  \label{eq:nonconservative}
\ee
%
which is not the same as Eq.~\ref{eq:conservative}. It is non-conservative, 
and we need a {\em correction term} to make it conservative. 

\noindent
In order to estimate the correction term, let's develop 
Eq.~\ref{eq:conservative}:
%
\be 
    \rho \frac{\p \phi}{\p t} 
  + \phi \frac{\p \rho}{\p t} 
  + {\bf u} \phi \nabla ( \rho ) 
  + \rho \phi \nabla ( {\bf u} ) 
  + \rho {\bf u} \nabla ( \phi ) 
  = D + S.
  \label{eq:conservative_develop_1}
\ee
%
Note that 3$^{rd}$ and 4$^{th}$ term on the left can be grouped together:
%
\be 
    {\bf u} \phi \nabla ( \rho ) 
  + \rho \phi \nabla ( {\bf u} ) 
  = \phi \nabla(\rho {\bf u}),
  \label{eq:note_1}
\ee
%
and the Eq.~\ref{eq:conservative_develop_1} can be written as:
%
\be 
    \rho \frac{\p \phi}{\p t} 
  + \phi \left[ 
       \frac{\p \rho}{\p t} 
     + \nabla ( \rho {\bf u} )
    \right]
  + \rho {\bf u} \nabla ( \phi ) 
  = D + S.
  \label{eq:conservative_develop_2}
\ee
%
Furhtermore, since mass conservation equation states that:
%
\be 
    \frac{\p \rho}{\p t}  
  + \nabla(\rho {\bf u})
  = 0,
  \label{eq:note_2}
\ee
%
and the Eq.~\ref{eq:conservative_develop_2} reduces to:
%
\be 
    \rho \frac{\p \phi}{\p t} 
  + \rho {\bf u} \nabla ( \phi ) 
  = D + S.
  \label{eq:conservative_develop_3}
\ee
%
The second term on the left can be written as:
%
\be 
    \rho {\bf u} \nabla ( \phi ) 
  = \rho \nabla ( {\bf u} \phi )
  - \rho \phi \nabla ( {\bf u} ),
  \label{eq:note_3}
\ee
%
which, when introduced in Eq.~\ref{eq:conservative_develop_3} gives the final
form of the governing equation:
%
\be 
    \rho \frac{\p \phi}{\p t} 
  + \rho \nabla ( {\bf u} \phi )
  - \rho \phi \nabla ( {\bf u} )
  = D + S,
  \label{eq:conservative_develop_4}
\ee
%
or:
%
\be 
    \rho \frac{\p \phi}{\p t} 
  + \rho \nabla ( {\bf u} \phi )
  = D + S
  + \rho \phi \nabla ( {\bf u} ).
  \label{eq:conservative_final}
\ee
%
The last term is the so-called {\em correction term} and must be added to 
discretization procedure to ensure conservatism.

\end{document}             % End of document. 
  
%%%%%%%%%%%%%%%%%%%%%%%%%%%%%%%%%%%%%%%%%%%%%%%%%%%%%%%%%%%%%%%%%%%%%%%%%%%%%%%%
% '$Id: conservative.tex,v 1.1 2011/05/30 11:50:05 niceno Exp $'/
%%%%%%%%%%%%%%%%%%%%%%%%%%%%%%%%%%%%%%%%%%%%%%%%%%%%%%%%%%%%%%%%%%%%%%%%%%%%%%%%
