\documentclass{report} 

%-------------------------
% begin and end equations
%-------------------------
\newcommand{\be} {\begin{equation}}
\newcommand{\ee} {\end{equation}}
\newcommand{\bea}{\begin{eqnarray}}
\newcommand{\eea}{\end{eqnarray}}
\newcommand{\ba} {\begin{array}}
\newcommand{\ea} {\end{array}}

%-----------------
% text formatting 
%-----------------
\newcommand{\noi}{\noindent}

%---------------
% special signs
%---------------
\newcommand{\p}  {\partial}
\newcommand{\s}  {\star}
\newcommand{\Dx} {\Delta}

%-----------------
% velocity vector
%-----------------
\newcommand{\uvw}{{\bf u}}          

%---------------------------
% putting things up or down
%---------------------------
\newcommand{\ol} {\overline}
\newcommand{\ul} {\underline}

%---------------------
% logical coordinates
%---------------------
\newcommand{\ijk}     { {i,j,k} }      % central

\newcommand{\ipjk}    { {i+1,j,k} }    % i+1
\newcommand{\imjk}    { {i-1,j,k} }    % i-1
\newcommand{\ijpk}    { {i,j+1,k} }    % j+1
\newcommand{\ijmk}    { {i,j-1,k} }    % j-1
\newcommand{\ijkp}    { {i,j,k+1} }    % k+1
\newcommand{\ijkm}    { {i,j,k-1} }    % k-1

\newcommand{\ijpkp}   { {i,j+1,k+1} }  % i
\newcommand{\ijpkm}   { {i,j+1,k-1} }
\newcommand{\ijmkp}   { {i,j-1,k+1} }
\newcommand{\ijmkm}   { {i,j-1,k-1} }

\newcommand{\ipjkp}   { {i+1,j,k+1} }  % j
\newcommand{\ipjkm}   { {i+1,j,k-1} }
\newcommand{\imjkp}   { {i-1,j,k+1} }
\newcommand{\imjkm}   { {i-1,j,k-1} }

\newcommand{\ipjpk}   { {i+1,j+1,k} }  % k
\newcommand{\ipjmk}   { {i+1,j-1,k} }
\newcommand{\imjpk}   { {i-1,j+1,k} }
\newcommand{\imjmk}   { {i-1,j-1,k} }

%--------------------------------------
% compute length for framing equations
%--------------------------------------
\newlength{\ii}
\setlength{\ii}{\textwidth}
\addtolength{\ii}{-2\fboxsep}
\addtolength{\ii}{-2\fboxrule}

% To frame an equation, use:
% \begin{center} \fbox{ \begin{minipage}{\ii} \be
% ... equation ... 
% \ee \end{minipage} } \end{center}

%-------------------
% to include figures
%-------------------
\usepackage{graphicx}
\usepackage{psfrag}
% \newcommand{\thickbox}[2]{\thicklines\put(0,0){\framebox(#1,#2){}}}
\newcommand{\thickbox}[2]{}

%----------------------------
% fancy mathematical symbols
%----------------------------
\usepackage{amssymb}

%---------------
% PSI-Boil sign
%---------------
\newcommand{\psiboil}{\sffamily PSI-Boil}


\begin{document}  

%%%%%%%%%%%%%%%%%%%%
%                  %
%  Wall treatment  %
%                  %
%%%%%%%%%%%%%%%%%%%%
\section{Wall treatment}

This document describes details of implementation of the power-law wall function.
In viscous sub-layer ($\delta^+ < \delta_C^+$):
%
\be
  u^+ = \delta^+
  \label{eq:u_plus_viscous}
\ee
%
In fully developed region ($\delta^+ >= \delta_C^+$):
% 
\be
  u^+ = A(\delta^+)^B
  \label{eq:u_plus_turbulent}
\ee
%
with $A=8.3$ and $B=1/7$. Here, $u^+$ and $\delta^+$ are non-dimensional velocity 
and non-dimensional wall distance respectivelly, defined as:
%
\be
  u^+ = \frac{u}{u_{\tau}}
  \label{eq:u_plus}
\ee
%
\be
  \delta^+ = \frac{ u_{\tau} \delta }{\nu}
  \label{eq:delta_plus}
\ee
%
where $\delta$ is the wall-distance and $u_{\tau}$ friction velocity defined as:
%
\be
  u_{\tau} = \sqrt{ \frac{\tau_w}{\rho} }
  \label{eq:u_tau}
\ee
%
where $\tau_w$ is friction at the wall, which is usually cumbersome to evaluate 
in a numerical simulation, but is needed for the wall function. Therefore, we 
try to express law of the wall in terms of know (computed) quantities.

%%%%%%%%%%%%%%%%%%%%%%%
%  Viscous sub-layer  %
%%%%%%%%%%%%%%%%%%%%%%%
\subsection{Viscous sub-layer}

In viscous sub-layer, inserting equations~(\ref{eq:u_plus}) and~(\ref{eq:delta_plus})
into~(\ref{eq:u_plus_viscous}):
%
\be
  \frac{u}{u_\tau} = \frac{ u_{\tau} \delta }{\nu}
\ee
%
leading to:
%
\be
  u_\tau = \sqrt{ \frac{u \nu}{\delta} }
  \label{eq:u_tau_viscous}
\ee
%
and from equation~(\ref{eq:delta_plus}):
%
\be
  \fbox {$
  \delta^+ = \sqrt{ \frac{u \delta}{\nu} }
  $ }
  \label{eq:delta_plus_viscous}
\ee
%
Friction at the wall can be estimated from equations~(\ref{eq:u_tau}) 
and~(\ref{eq:u_tau_viscous}) as:
%
\be
  \fbox {$
  \tau_w = \rho \frac{u \nu}{\delta}
  $ }
  \label{eq:tau_w_viscous}
\ee

\subsection{Turbulent layer}

In fully developed region, inserting equations~(\ref{eq:u_plus}) and~(\ref{eq:delta_plus}) into
the equation for power-law~(\ref{eq:u_plus_turbulent}), we have:
%
\be
  \frac{u}{u_\tau} = A \left( \frac{ u_{\tau} \delta }{\nu} \right)^B
\ee
%
leading to:
%
\be
  \frac{u}{A} \left( \frac{\nu}{\delta} \right) ^ B = u_\tau^{(1+B)}
\ee
%
so $u_\tau$ can be evaluated from:
%
\be
  u_\tau = \left[ 
              \frac{u}{A} \left( \frac{\nu}{\delta} \right) ^ B 
           \right] ^ \frac{1}{1+B}
  \label{eq:u_tau_turbulent}
\ee
%
From equation~(\ref{eq:delta_plus}), we get:
%
\bea
  \delta^+ & = & \frac{\delta}{\nu} 
            \left[ 
                   \frac{u}{A} \left( \frac{\nu}{\delta} \right) ^ B 
            \right] ^ \frac{1}{1+B}                     \nonumber \\
      & = & \left[ 
                   \frac{u}{A} 
                   \left( \frac{\nu}{\delta} \right) ^ B 
                   \left( \frac{\delta}{\nu} \right) ^ {1+B} 
            \right] ^ \frac{1}{1+B}                     \nonumber \\
      & = & \left[ 
                   \frac{u}{A} 
                   \left( \frac{\delta}{\nu} \right) ^ {-B} 
                   \left( \frac{\delta}{\nu} \right) ^ {1+B} 
            \right] ^ \frac{1}{1+B}                     \nonumber \\
      & = & \left[ 
                   \frac{u}{A} 
                   \left( \frac{\delta}{\nu} \right) ^ {-B} 
                   \left( \frac{\delta}{\nu} \right) ^ {1+B} 
            \right] ^ \frac{1}{1+B}                     
\eea
\be
  \fbox
  { $
  \delta^+ = \left( 
                   \frac{u}{A} \frac{\delta}{\nu} 
            \right) ^ \frac{1}{1+B}                               
  $ }
  \label{eq:delta_plus_turbulent}
\ee
%
Friction at the wall can be estimated from equations~(\ref{eq:u_tau}) 
and~(\ref{eq:u_tau_turbulent}) as:
%
\be
  \fbox
  { $
  \tau_w = \rho \left[ 
                   \frac{u}{A} \left( \frac{\nu}{\delta} \right) ^ B 
                \right] ^ \frac{2}{1+B}
  $ }
  \label{eq:tau_w_turbulent}
\ee
%

%%%%%%%%%%%%%%%%%%
%  An algorithm  %
%%%%%%%%%%%%%%%%%%
\subsection{An algorithm}

A sample algorithm for estimation of the wall friction, might be assembled
as follows:

\begin{itemize}

   \item Estimate $\delta^+$ from linear law (assume you are in viscous sub-layer)

   \item if $\delta^+ < \delta_C^+$, evaluate $tau_w$ from equation~(\ref{eq:tau_w_viscous}).

   \item otherwise, evaluate $tau_w$ from equation~(\ref{eq:tau_w_turbulent}) and 
         recompute $\delta^+$ from~(\ref{eq:delta_plus_turbulent}).

\end{itemize}

%%%%%%%%%%%%%%%%%%%%%%%
%  Critical distance  %
%%%%%%%%%%%%%%%%%%%%%%%
\subsection{Critical distance}

We refer to a boundary between viscous and turbulent region as critical 
distance and denote it with~$\delta_C^+$. We can get it's value, if we
equate linear (\ref{eq:u_plus_viscous}) and power (\ref{eq:u_plus_turbulent})
laws of the wall, namely:
%
\be
  \delta_C^+ = A (\delta_C^+) ^ B
\ee
%
\be
  \delta_C^+ = A ^ {\frac{1}{1-B}}
\ee
%
if $A=8.3$ and $B=1/7$, $\delta_C^+ = 11.81021420062548881$.

\end{document}
